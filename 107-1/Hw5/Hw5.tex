\documentclass[a4paper,11pt]{article}
\usepackage[top=2cm,bottom=2cm,outer=2cm,inner=2cm]{geometry}
\usepackage[utf8]{inputenc}
\usepackage[T1]{fontenc}
\usepackage[inline]{enumitem}
\usepackage{amsfonts}
\usepackage{amsmath}


\title{Real Analysis \\ Homework 5}
\author{Yueh-Chou Lee}
\date{\today}
\begin{document}
\maketitle
\begin{enumerate}


\item (Exercise 5.3)\\
 Let $\{ f_k \}$ be a sequence of nonnegative measurable functions defined on $E$. If $f_k \to f$ and $f_k \leq f$ a.e. on $E$, show that $\int_{E} f_k \to \int_{E} f$.\\
 \newline
 \textit{\textbf {Proof.}}\\
 Since $f_k \to f$ a.e. and measurable in $E$, then $f$ is also measurable.\\
 By Lebesgue Dominated Convergence Theorem for Nonnegative Functions, since\\$0 \leq f_k$, $f_k \leq f$ $\forall k$ with $\int_E f dx < +\infty$ and $f_k \to f$ a.e. in $E$, then $\int_E f_k(x) dx \to \int_E f(x) dx$.\\








\item (Exercise 5.4)\\
 If $f \in L(0,1)$, show that $x^k f(x)$ in $L(0,1)$ for $k = 1,2,...$, and that $\int_{0}^{1} x^k f(x) dx \to 0$.\\
 \newline
 \textit{\textbf {Proof.}}\\
 Since $f \in L(0,1)$ and $x \in (0,1)$, then $|f|$ is also measurable and $|x^k f(x)| \leq |f(x)|$ in $(0,1)$.\\
 $x^k f(x) \to 0$ a.e. as $k \to 0$.\\
 By Lebesgue Dominated Convergence Theorem, since $x^k f(x) \to 0$ a.e. in $(0,1)$,\\$|x^k f(x)| \leq |f(x)|$ $\forall k$ and $|f|$ is also measurable, then $\int_{(0,1)} f_k(x) dx \to \int_{(0,1)} 0 dx = 0$.\\ 








\item (Exercise 5.5)\\
 Use Egorov's theorem to prove the bounded convergence theorem.\
 
 \textbf{Recall (Egorov's Theorem):}\\
 Suppose that $\{ f_k \}$ is a sequence of measurable functions that converges a.e. in a set $E$ of finite measure to a finite limit $f$. Then given $\epsilon > 0$, there is a closed subset $F$ of $E$ such that $|E - F| < \epsilon$ and $\{ f_k \}$ converge uniformly to $F$.\

 \textbf{Recall (Bounded Convergence Theorem):}\\
 Let $\{ f_k \}$ be a sequence of measurable functions on $E$ such that $f_k \to f$ a.e. in $E$. If $|E| < +\infty$ and there is a finite constant $M$ such that $|f_k| \leq M$ a.e. in $E$, then $\int_E f_k \to \int_E f$.\\
 \newline
 \textit{\textbf {Proof.}}\\
 By Egorov's theorem, for any $\epsilon$, there exists a closed set $F \subseteq E$ such that $\{ f_k \}$ converges uniformly on $F$ and $|E - F| < \frac{M \epsilon}{4}$.\\
 Since $|f_k| \leq M$ a.e. and $M|E| < \infty$, by Fatou's lemma, we have
 $$\begin{aligned}
 \int_F f
 &= \int_F \underset{k \to \infty}{\lim \inf} f_k\\
 &\leq \underset{k \to \infty}{\lim \inf} \int_F f_k\\
 &\leq \underset{k \to \infty}{\lim \sup} \int_F f_K\\
 &\leq \int_F \underset{k \to \infty}{\lim \sup} f_k\\
 &= \int_F f
 \end{aligned}$$
 Then $\int_F f_k \to \int_F f$.\\
 There exists $N > 0$ such that for all $k \geq N$, we have $\left| \int_F f - \int_F f_k \right| < \frac{\epsilon}{2}$.\\
 Hence, for $k \geq N$
 $$\left| \int_E f - \int_E f_k \right|
 \leq \left| \int_F f - \int_F f_k \right| + \left| \int_{E - F} f \right| + \left| \int_{E - F} f_k \right| < \epsilon$$
 Then $\int_E f_k \to \int_E f$.\\
 






\item (Exercise 5.6)\\
 Let $f(x,y)$, $0 \leq x, y \leq 1$, satisfy the following conditions: for each $x, f(x,y)$ is an integrable function of $y$, and $(\partial f(x,y) / \partial x)$ is a bounded function of $(x,y)$. Show that $(\partial f(x,y) / \partial x)$ is a measurable function of $y$ for each $x$ and
 $$\frac{d}{dx} \int_{0}^{1} f(x,y) dy = \int_{0}^{1} \frac{\partial}{ \partial x} f(x, y) dy$$
 \newline
 \textit{\textbf {Proof.}}

 \begin{enumerate}
 \item $(\partial f(x,y) / \partial x)$ is a measurable function of $y$ for each $x$:\\
 By definition, we know for every $x$
 $$\frac{\partial f(x,y)}{ \partial x} = \underset{h \to 0}{\lim} \frac{f(x+h,y) - f(x,y)}{h}$$
 Since $f(x,y)$ is an integrable function of $y$ for every $x$, $f(x,y)$ is measurable function of $y$ for every $x$, then $\frac{\partial f(x,y)}{ \partial x}$ is also measurable for every $x$.\\

 \item
 $$\begin{aligned}
 \frac{d}{dx} \int_{0}^{1} f(x,y) dy
 &= \underset{h \to 0}{\lim} \frac{\int_0^1 f(x+h,y)dy - \int_0^1 f(x,y) dy}{h}\\
 &= \underset{h \to 0}{\lim} \int_0^1 \frac{f(x+h,y) - f(x,y)}{h} dy\\
 \end{aligned}$$
 By Mean Value Theorem, there exists $0 < h' \leq h$ such that
 $$\frac{f(x+h,y) - f(x,y)}{h} = \frac{\partial}{\partial x} f(x+h',y)$$
 which is a bounded function of $(x,y)$, then by Bounded Convergence Theorem
 $$\frac{d}{dx} \int_{0}^{1} f(x,y) dy = \int_{0}^{1} \frac{\partial}{ \partial x} f(x, y) dy$$\

 \end{enumerate}








\item (Exercise 5.7)\\
 Give an example of an $f$ that is not integrable, but whose improper Riemann integral exists and is finite.\\
\newline
\textit{\textbf {Proof.}}\\
 Let $f$ be a function on $[1, \infty)$ with $f(x) = (-1)^n \frac{1}{n}$ if $x \in [n,n+1)$ where $n \in \mathbb{Z}^+$, then
 $$\int_{[1,\infty)} f^+ = \int_{[1,\infty)} \max{\{ f, 0\}} =\sum_{k=1}^{\infty} \frac{1}{2k} |[2k, 2k+1)| = \infty$$
 and
 $$\int_{[1,\infty)} f^- = \int_{[1,\infty)} -\min{\{ f, 0\}} =\sum_{k=1}^{\infty} \frac{1}{2k-1} |[2k-1, 2k)| = \infty$$
 $f$ is said to be integrable in $[1,\infty)$\\
 $\iff$ $|\int_[1,\infty) f(x) dx| = |\int_[1,\infty) f^+(x) dx - \int_[1,\infty) f^-(x) dx| < \infty$.\\

 Since $\int_{[1,\infty)} f^+(x) dx = \infty$ and $\int_{[1,\infty)} f^-(x) dx = \infty$, hence, $f$ is not integrable.\\

 But
 $$(R) \int_{[1,\infty)} f(x) dx = \sum_{n = 1}^{\infty} (-1)^n \frac{1}{n} < \infty$$
 It implies that $f$ is Riemann integrable.\\









\item (Exercise 5.9)\\
 If $p > 0$ and $\int_{E} |f - f_k|^p \to 0$ as $k \to \infty$, show that $f_k \overset{m}{\to} f$ on $E$ (and thus that there is a subsequence $f_{k_j} \to f$ a.e. in $E$).\\
 \newline
 \textit{\textbf {Proof.}}\\
 Let $\omega(\alpha) = |\{ x \in E : f(x) > \alpha \}|$ where $\alpha > 0$.\\
 We first need to prove that $\omega(\alpha) \leq \frac{1}{\alpha^p} \int_{ \{f > \alpha\} } f^p(x) dx$.\\
 Let $g(x) = \left\{ \begin{array}{ll}
 					\alpha, & \mbox{if $f(x) > \alpha$} \\
 					0, & \mbox{o.w.}
 					\end{array}
 					\right.$
 Then
 $$\int_{\{ f > \alpha \}} f^p \geq \int_{\{ f > \alpha \}} g^p = \int_{\{ f > \alpha \}} \alpha^p = \alpha^p |\{ f > \alpha \}| = \alpha^p \omega(\alpha)$$
 Hence, 
 $$\omega(\alpha) \leq \frac{1}{\alpha^p} \int_{ \{f > \alpha\} } f^p(x) dx$$\\

 Now, we let
 $$\omega'(\alpha) = |\{ x \in E: |f(x) - f_k(x)|^p > \alpha \}|$$
 By above, we then have
 $$\omega'(\alpha) = |\{ x \in E: |f(x) - f_k(x)|^p > \alpha \}| \leq \frac{1}{\alpha^p} \int_{E} |f - f_k|^p$$
 That is
 $$|\{ x \in E: |f(x) - f_k(x)| > \alpha^{1/p} \}| \leq \frac{1}{\alpha^p} \int_{E} |f - f_k|^p$$
 Hence,
 $$0 \leq \underset{k \to \infty}{\lim} |\{ x \in E: |f(x) - f_k(x)| > \alpha^{1/p} \}| \leq \frac{1}{\alpha^p} \cdot \underset{k \to \infty}{\lim} \int_{E} |f - f_k|^p = 0$$
 Thus,
 $$\underset{k \to \infty}{\lim} |\{ x \in E: |f(x) - f_k(x)| > \alpha^{1/p} \}| = 0$$
 Since $\alpha^{1/p}$ can be any positive real number, we have that $f_k \overset{m}{\to} f$.\\








\item (Exercise 5.10)\\
 If $p > 0$, $\int_{E} |f - f_k|^p \to 0$ and $\int_{E} |f_k|^p \leq M$ for all $k$, show that $\int_{E} |f|^p \leq M$.\\
 \newline
 \textit{\textbf {Proof.}}\\
 By Exercise 5.9, since $\int_{E} |f - f_k|^p \to 0$, $\forall p>0$, then $f_k \overset{m}{\to} f$ on $E$.\\
 So we can find the subsequence $\{ f_{k_j} \}$ such that $f_{k_j} \to f$ a.e. in $E$.\\
 Then $|f_{k_j}|^p \to |f|^p$ a.e. in $E$.\\
 By Fatou's Lemma, we have
 $$\int_{E} |f|^p = \int_{E} \underset{j \to \infty}{\lim \inf} |f_{k_j}|^p \leq \underset{j \to \infty}{\lim \inf} \int_{E} |f_{k_j}|^p \leq \underset{j \to \infty}{\lim \inf} M = M$$\\








\item (Exercise 5.13)
 \begin{enumerate}
 \item Let $\{ f_k \}$ be a sequence of measurable functions on $E$. Show that $\sum f_k$ converges absolutely a.e. in $E$ if $\sum \int_{E} |f_k| < +\infty$. (Use Theorem 5.16 and 5.22.)
 \item If $\{ r_k \}$ denotes the rational numbers in $[0,1]$ and $\{ a_k \}$ satisfies $\sum |a_k| < +\infty$, show that $\sum a_k |x - r_k|^{-1/2}$ converges absolutely a.e. in $[0,1]$.\
 \end{enumerate}
 \textbf{Recall (Theorem 5.16):}\\
 If $f_k$, $k = 1,2,...,$ are nonnegative and measurable, then
 $$\int_{E} \left( \sum_{k=1}^{\infty} f_k \right) = \sum_{k=1}^{\infty} \int_{E} f_k$$\\
 \textbf{Recall (Theorem 5.22):}\\
 If $f \in L(E)$, then $f$ is finite a.e. in $E$.\\
 
 \textit{\textbf {Proof.}}\\
 \begin{enumerate}
 \item
 If $\int_{E} \left| \sum f_k \right| < \infty$, then $\sum f_k$ converges absolutely a.e. in $E$.\\
 $$\int_{E} \left| \sum f_k \right| = \int_{E} \sum |f_k|$$
 $|f_k|$ is measurable on $E$, since $f_k$ is measurable on $E$.\\
 By Theorem 5.16, since $|f_k| \geq 0$ and measurable on $E$, then
 $$\int_{E} \left| \sum_{k=1}^{\infty} f_k \right| = \int_{E} \sum_{k=1}^{\infty} |f_k| = \sum_{k=1}^{\infty} \int_{E} |f_k| < +\infty$$
 Hence, $\sum f_k$ converges absolutely a.e. in $E$.\\

 \item
 If $\int_{[0,1]} \left|\sum a_k |x-r_k|^{-1/2} \right| dx < \infty$, then $\sum a_k |x - r_k|^{-1/2}$ converges absolutely a.e. in $[0,1]$.\\
 $$\begin{aligned}
 \int_{[0,1]} \left| \sum a_k |x-r_k|^{-1/2} \right| dx
 &\leq \int_{[0,1]} \sum |a_k| |x-r_k|^{-1/2} dx\\
 &= \sum \int_{[0,1]} |a_k| |x-r_k|^{-1/2} dx\\
 &= \sum |a_k| \int_{[0,1]} |x-r_k|^{-1/2} dx\\
 &= \sum |a_k| (2r_k^{1/2} + 2(1-r_k)^{1/2}) dx\\
 &< \infty
 \end{aligned}$$
 Hence, $\sum a_k |x - r_k|^{-1/2}$ converges absolutely a.e. in $[0,1]$.\\

 \end{enumerate}

\end{enumerate}
\end{document}




