\documentclass[a4paper,11pt]{article}
\usepackage[top=2cm,bottom=2cm,outer=2cm,inner=2cm]{geometry}
\usepackage[utf8]{inputenc}
\usepackage[T1]{fontenc}
\usepackage[inline]{enumitem}
\usepackage{amsfonts}
\usepackage{amsmath}


\title{Real Analysis \\ Homework 7}
\author{Yueh-Chou Lee}
\date{\today}
\begin{document}
\maketitle
 \begin{enumerate}
 	\item (Exercise 6.1)
 		\begin{enumerate}
 			\item [(a)] Let $E$ be a measurable subset of $\mathbb{R}^2$ such that for almost every $x \in \mathbb{R}^1$, $\{ y: (x,y) \in E \}$ has $\mathbb{R}^1$-measure zero. Show that $E$ has measure zero and that for almost every $y \in \mathbb{R}^1$, $\{x:(x,y) \in E \}$ has measure zero.

 			\item [(b)] Let $f(x,y)$ be nonnegative and measurable in $\mathbb{R}^2$. Suppose that for almost every $x \in \mathbb{R}^1$, $f(x,y)$ is finite for almost every $y$. Show that for almost every $y in \mathbb{R}^1$, $f(x,y)$ is finite for almost every $x$.\\
 		\end{enumerate}

 		\textit{\textbf {Proof.}}

 		\begin{enumerate}
 			\item [(a)] Since $\chi_E (x,y)$ is nonnegative, measurable in $\mathbb{R}^2$ ($E$ is a measurable subset of $\mathbb{R}^2$) and $\{ y : (x,y) \in E \}$ has $\mathbb{R}^1$-measure zero, $\int_{\mathbb{R}^1} \chi_E (x,y) dx = 0$, by Tonelli's Theorem, we have

 			$$\begin{aligned}
 			|E|
 			&= \int \int_{\mathbb{R}^2} \chi_E (x,y) dx dy\\
 			&= \int_{\mathbb{R}^1} \left[ \int_{\mathbb{R}^1} \chi_E (x,y) dy \right] dx\\
 			&= \int_{\mathbb{R}^1} \left| \{ y : (x,y) \in E \} \right| dx\\
 			&= 0
 			\end{aligned}$$

 			So $|E|$ has measure zero.\\

 			$$\begin{aligned}
 			|E|
 			&= \int \int_{\mathbb{R}^2} \chi_E (x,y) dx dy\\
 			&= \int_{\mathbb{R}^1} \left[ \int_{\mathbb{R}^1} \chi_E (x,y) dx \right] dy\\
 			&= \int_{\mathbb{R}^1} \left| \{ x : (x,y) \in E \} \right| dy\\
 			&= 0
 			\end{aligned}$$

 			So $\{ x : (x,y) \in E \}$ has measure zero almost every $y$.\\

 			\item [(b)] Since for almost every $x \in \mathbb{R}^1$, $f(x,y)$ is finite for almost every $y$, then $\{ y | f(x,y) = \infty \}$ has measure zero.\\
 			Let $Z = \{ (x,y) | f(x,y)=\infty \}$, $Z_1=\{ x | f(x,y)=\infty \}$ and $Z_2 = \{ y | f(x,y)=\infty \}$, then $Z = Z_1 \times Z_2$.\\
 			Since $f(x,y)$ is nonnegative function and measurable in $\mathbb{R}^2$, $\int_{Z_2} dy = |Z_2| = 0$, by Tonelli's theorem, we have

 			$$\int \int_Z dx dy
 			= \int_{Z_2} \left[ \int_{Z_1} dx \right] dy
 			= \int_{Z_1} \left[ \int_{Z_2} dy \right] dx
 			= 0$$

 			Hence $\int_{Z_1} dx = 0$ for almost every $y$, then $Z_1=\{ x | f(x,y)=\infty \}$ has also measure zero.\\
 			So $f(x,y)$ is finite for almost every $x$.\\


 		\end{enumerate}









 	\item (Exercise 6.3)\\
 		Let $f$ be measurable and finite a.e. on $[0,1]$. If $f(x) - f(y)$ is integrable over the square $0 \leq x \leq 1, 0 \leq y \leq 1$, show that $f \in L[0,1]$.\\
 		
 		\textit{\textbf {Proof.}}\\

 		Let $I_1 = (0,1)$ and $I_2 = (0,2)$ such that $I = I_1 \times I_2$.\\
 		Since $g(x,y) = f(x) - f(y) \in L(I)$ , by Fubini's Theorem, we know that for almost every $x \in I_1$, $g(x,y)$ is measurable and integrable on $I_2$ as a function of $y$.\\
 		Pick any $x_0 \in (0,1)$ then $g(x_0,y) = f(x_0) - f(y)$ is measurable and integrable on $I_2$, that is $f(y)$ is integrable on $(0,1)$.\\
 		Hence $f \in L(I_2) = L(0,1)$.\\




	


	\item (Exercise 6.5)
		\begin{enumerate}
			\item [(a)] If $f$ is nonnegative and measurable on $E$ and $\omega(y) = |\{ x \in E: f(x) > y \}|, y > 0$, use Tonelli's theroem to prove that $\int_E f = \int_0^\infty \omega(y) dy$. (By definition of the integral, $\int_E f = |R(f,E)| = \int \int_{R(f,E)} dx dy$. Use the observation in the proof of Theroem 6.11 that $\{ x \in E: f(x) \geq y \} = \{ x: (x,y) \in R(f,E) \}$, and recall that $\omega(y) = |\{ x \in E: f(x) \geq y \}|$ unless $y$ is a point of discontinuity of $\omega$. )

			\item [(b)] Deduce from this special case the general formula
			$$\int_E f^p = p \int_0^\infty y^{p-1} \omega(y) dy \hspace{0.4cm} \left( f \geq 0, \hspace{0.1cm} 0 < p < \infty \right)$$
		\end{enumerate}

 		\textit{\textbf {Proof.}}

 		\begin{enumerate}
 			\item [(a)] By definition of the integral and using the observation in the proof of Theroem 6.11 that $\{ x \in E: f(x) \geq y \} = \{ x: (x,y) \in R(f,E) \}$, we have
 			$$\begin{aligned}
 			\int_E f
 			&= |R(f,E)| = \int \int_{R(f,E)} dx dy\\
 			&= \int_0^\infty \left[ \int_{\{ x: (x,y) \in R(f,E) \}} dx \right] dy\\
 			&= \int_0^\infty \left[ \int_0^\infty \chi_{\{ x \in E: f(x) \geq y \}} dx \right] dy\\
 			&= \int_0^\infty \omega(y) dy
 			\end{aligned}$$



 			\item [(b)] The truth that
 			$$f^p(x) = \int_0^{f(x)} p \cdot y^{p-1} \hspace{0.1cm} dy$$
 			for all $x \in E$.\\

 			By using the result of part (a), Tonelli's Theorem and the above truth, then we have

 			$$\begin{aligned}
 			\int_E f^p(x) dx
 			&= \int_E \int_0^{f(x)} p \cdot y^{p-1} \hspace{0.1cm} dy \hspace{0.1cm} dx\\
 			&= \underset{R(f,E)}{\int \int} p \cdot y^{p-1} \hspace{0.1cm} dy \hspace{0.1cm} dx\\
 			&= \int_0^\infty \left[ \int_{\{x \in E: f(x) \geq y \}} p \cdot y^{p-1} \hspace{0.1cm} dx \right] dy\\
 			&= p \int_0^\infty y^{p-1} \left[ \int_{\{x \in E: f(x) \geq y \}} \hspace{0.1cm} dx \right] dy\\
 			&= p \int_0^\infty y^{p-1} \omega(y) \hspace{0.1cm} dy
 			\end{aligned}$$\\
 		\end{enumerate}






	\item (Exercise 6.10)\\
		Let $v_n$ be the volume of the unit ball in $\mathbb{R}^n$. Show by using Fubini's theroem that
		$$v_n = 2v_{n-1} \int_0^1 \left( 1 - t^2 \right)^{(n-1)/2} dt$$
		(We also observe that by setting $w = t^2$, the integral is a multiple of a classical $\beta$-function and so can be expressed in terms of the $\Gamma$-function: $\Gamma(s) = \int_0^\infty e^{-t} t^{s-1} dt, \hspace{0.1cm} s > 0$.)\\
	
 		\textit{\textbf {Proof.}}\\

 		Using the induction to prove this formula.\\
 		Let $v_1 = 2$, that is the length of the interval $[-1,1]$.\\
 		If $n = 2$, $v_2$ will be the area of the unit circle, then $v_2 = \pi$. Moreover
 		$$2v_1 \int_0^1 (1 - t^2)^{1/2} \hspace{0.1cm} dt = 2 \cdot 2 \cdot \frac{\pi}{4} = \pi = v_2$$
 		So it's ture when $n = 2$.\\
 		Suppose the formula holds for $n-1$ and let
 		$$B_n = \{ x \in \mathbb{R}^n : x_1^2 + ... + x_n^2 \leq 1 \}$$
 		be the unit ball in $\mathbb{R}^n$.\\
 		Using Tonelli's Theorem, then we have
 		$$\begin{aligned}
 		v_n
 		&= \int ... \int_{B_n} 1\\
 		&= \int ... \int_{\{ x_1^2 + ... + x_n^2 \leq 1 \}} 1 \hspace{0.1cm} dx_1 \hspace{0.1cm} ... \hspace{0.1cm} dx_n\\
 		&= \int_{-1}^{1}  \left( \int ... \int_{\{ x_2^2 + ... x_n^2 \leq 1 - x_1^2 \}} 1 \hspace{0.1cm} dx_2 \hspace{0.1cm} ... \hspace{0.1cm} dx_n \right) \hspace{0.1cm} dx_1
 		\end{aligned}$$

 		Let $u_j = \frac{x_j}{\sqrt{1 - x_1^2}}$ for $j = 2,...,n$, then $\frac{d u_j}{dx_j} = \frac{1}{\sqrt{1 - x_1^2}}$.\\
 		Hence
 		$$\begin{aligned}
 		v_n
 		&= \int_{-1}^{1}  \left( \int ... \int_{\{ x_2^2 + ... x_n^2 \leq 1 - x_1^2 \}} 1 \hspace{0.1cm} dx_2 \hspace{0.1cm} ... \hspace{0.1cm} dx_n \right) \hspace{0.1cm} dx_1\\
 		&= \int_{-1}^{1} \left( \int ... \int_{\{ u_1^2 + ... + u_n^2 \leq 1 \}} (1 - x_1^2)^{\frac{n-1}{2}} \hspace{0.1cm} du_2 \hspace{0.1cm} ... \hspace{0.1cm} du_n \right) \hspace{0.1cm} dx_1\\
 		&= \int_{-1}^{1} \left( \int ... \int_{\{ u_1^2 + ... + u_n^2 \leq 1 \}}  \hspace{0.1cm} du_2 \hspace{0.1cm} ... \hspace{0.1cm} du_n \right) (1 - x_1^2)^{\frac{n-1}{2}} \hspace{0.1cm} dx_1\\
 		&= \int_{-1}^{1} (v_{n-1}) (1 - x_1^2)^{\frac{n-1}{2}} \hspace{0.1cm} dx_1\\
 		&= v_{n-1} \int_{-1}^{1} (1 - x_1^2)^{\frac{n-1}{2}} \hspace{0.1cm} dx_1\\
 		&= 2v_{n-1} \int_{0}^{1} (1 - x_1^2)^{\frac{n-1}{2}} \hspace{0.1cm} dx_1\\
 		&= 2v_{n-1} \int_0^1 \left( 1 - t^2 \right)^{(n-1)/2} dt
 		\end{aligned}$$\






	\item (Exercise 6.11)\\
		Use Fubini's theorem to prove that
		$$\int_{\mathbb{R}^n} e^{-|x|^2} dx = \pi^{n/2}$$
		(For $n = 1$, write $\left( \int_{-\infty}^{+\infty} e^{-x^2} dx \right)^2 = \int_{-\infty}^{+\infty} \int_{-\infty}^{+\infty} e^{-x^2-y^2} dx dy$ and use polar coordinates. For $n > 1$, use the formula $e^{-|x|^2} = e^{-x_1^2} ... e^{-x_n^2}$ and Fubini's theorem to reduce to the case $n = 1$.)\\

 		\textit{\textbf {Proof.}}\\
 		By Fubini's Theorem, we know that
 		$$\begin{aligned}
 		\int_{\mathbb{R}^n} e^{|x|^2} dx
 		&= \int ... \int_{\mathbb{R}^n} e^{-x_1^2} ... e^{-x_n^2} dx_1...dx_n\\
 		&= \int_0^\infty \left[ \int ... \int_{\mathbb{R}^{n-1}} e^{-x_1^2} ... e^{-x_n^2} dx_2...dx_n \right] dx_1 \\
 		&= \int_0^\infty e^{-x_1^2} dx_1 \left[ \int ... \int_{\mathbb{R}^{n-1}} e^{-x_2^2} ... e^{-x_n^2} dx_2...dx_n \right]\\
 		&= ...\\
 		&= \int_0^\infty e^{-x_1^2} dx_1 \cdot ... \cdot \int_0^\infty e^{-x_n^2} dx_n\\
 		&= \sqrt{\pi} \cdot ... \cdot \sqrt{\pi}\\
 		&= \pi^{n/2}
 		\end{aligned}$$


 \end{enumerate}
\end{document}






