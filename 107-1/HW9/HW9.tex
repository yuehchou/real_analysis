\documentclass[a4paper,11pt]{article}
\usepackage[top=2cm,bottom=2cm,outer=2cm,inner=2cm]{geometry}
\usepackage[utf8]{inputenc}
\usepackage[T1]{fontenc}
\usepackage[inline]{enumitem}
\usepackage{amsfonts}
\usepackage{amsmath}


\title{Real Analysis \\ Homework 9}
\author{Yueh-Chou Lee}
\date{December 16, 2018}
\begin{document}
\maketitle
 \begin{enumerate}

 	\item (Exercise 7.4)\\
 		If $E_1$ and $E_2$ are measurable subsets of $\mathbb{R}^1$ with $|E_1| > 0$ and $|E_2| > 0$, prove that the set $\{ x: x = x_1 - x_2, x_1 \in E_1, x_2 \in E_2 \}$ contains an interval. (cf. Lemma 3.37.)\

 		\textit{\textbf {Proof.}}\\
 		Consider the set $-E_1$ and $E_2$, with positive finte measure, then both $\chi_{-E_1}$ and $\chi_{E_2}$ are integrable, and
 		$$\begin{aligned}
 		\int_x \chi_{-E_1} * \chi_{E_2} dx
 		&= \int_x \int_y \chi_{-E_1} (x-y) \chi_{E_2}(y) dy dx\\
 		&\overset{\mbox{Fubini}}{=} \int_y \int_x \chi_{-E_1} (x-y) \chi_{E_2}(y) dx dy\\
 		&= \int_y \chi_{E_2}(y) \int_x \chi_{-E_1} (x-y) dx dy\\
 		&= \int_y \chi_{E_2}(y) |-E_1| dy\\
 		&= |E_2||-E_1| > 0
 		\end{aligned}$$

 		So there must exist some point where $\chi_{-E_1} * \chi_{E_2} > 0$. Convolution is continuous, so $\chi_{-E_1} * \chi_{E_2} > 0$ on the interval, $(x_1, x_2)$, then for all $t \in (x_1, x_2)$,
 		$$\chi_{-E_1} * \chi_{E_2} (t) = \int_x \chi_{-E_1}(t-x) \chi_{E_2}(x) dx > 0$$
 		there must be some $x \in \mathbb{R}$ for which $\chi_{-E_1}(t-x) \chi_{E_2}(x) > 0$, then we have $x \in E_2$ and $t-x \in -E_1$,
 		$$t = (t - x) + (x) \in -E_1 + E_2 \Rightarrow t \in E_2 - E_1$$

 		Hence the set $\{ x: x = x_1 - x_2, x_1 \in E_1, x_2 \in E_2 \}$ contains an interval.\\







 	\item (Exercise 7.5)\\
 		Let $f$ be of bounded variation on $[a,b]$. If $f = g + h$, where $g$ is absolutely continuous and $h$ is singular, show that
 		$$\int_a^b \phi \hspace{0.1cm} df
 		= \int_a^b \phi f' \hspace{0.1cm} dx + \int_a^b \phi \hspace{0.1cm} dh$$
 		for any continuous $\phi$.\

 		\textit{\textbf {Proof.}}\\
 		$\int_a^b \phi dg$ exists because $g$ is absolutely continuous and therefore continuous.\\
 		$\int_a^b \phi df$ exists  because $f$ is of bounded variation.\\

 		Since $\int_a^b \phi dg$, $\int_a^b \phi df$ exist and
 		$$\int_a^b \phi df - \int_a^b \phi dg = \int_a^b \phi d(f-g) = \int_a^b \phi dh$$
 		then $\int_a^b \phi dh$ also exists.\\

 		By Theorem 7.32, so
 		$$\begin{aligned}
 		\int_a^b \phi df
 		&= \int_a^b \phi d(g+h)\\
 		&= \int_a^b \phi dg + \int_a^b \phi dh\\
 		&= \int_a^b \phi g' dx + \int_a^b \phi dh\\
 		&= \int_a^b \phi f' dx + \int_a^b \phi dh
 		\end{aligned}$$
 		where $\int_a^b \phi f' dx = \int_a^b \phi g' dx$ since $h$ is singular ($h' = 0$), then $f' = g'$ a.e.\\








 	\item (Exercise 7.8)\\
 		Prove  the following converse of Theorem 7.31: If $f$ is of bounded variation on $[a,b]$, and if the function $V(x) = V[a, x]$ is absolutely continuous on $[a,b]$, then $f$ is absolutely continuous on $[a,b]$.\

 		\textit{\textbf {Proof.}}\\
 		Since $f$ is of bounded variation on $[a,b]$ and the function $V(x) = V[a, x]$,
 		$$V(x) = V[a, x] \leq V[a,b] < \infty$$
 		for all $x \in [a,b]$.\\

 		Since $V(x)$ is absoultely continuous on $[a,b]$, for given $\epsilon > 0$, there exists $\delta > 0$ such that for any collection $\{[a_i,b_i]\}$ of nonoverlapping subintervals of $[a,b]$,
 		$$\sum |V(b_i) - V(a_i)| < \epsilon, \hspace{0.5cm} \mbox{if } \sum (b_i - a_i) < \delta$$
 		$V(b_i) - V(a_i)$ is well-defined since $V(x) < \infty$ for all $x \in [a,b]$, then if $\sum (b_i - a_i) < \delta$,
 		$$\begin{aligned}
 		\sum |f(b_i) - f(a_i)|
 		&\leq \sum V[a_i,b_i]\\
 		&= \sum |V[a,b_i] - V[a,a_i]|\\
 		&= \sum |V(b_i) - V(a_i)|\\
 		< \epsilon
 		\end{aligned}$$

 		Hence $f$ is absolutely continuous on $[a,b]$.\\









 	\item (Exercise 7.9)\\
 		If $f$ is of bounded variation on $[a,b]$, show that
 		$$\int_a^b |f'| \leq V[a,b]$$
 		Show that if equality holds in this inequality, then $f$ is absolutely continuous on $[a,b]$. (For the second part, use Theorems 2.2(ii) and 7.24 to show that $V(x)$ is absolutely continuous and then use the result of Exercise 8.)\

 		\textit{\textbf {Proof.}}\\
 		\begin{enumerate}
 			\item[(i)].\\
 				By Theorem 7.24, since $f$ is of bounded variation on $[a,b]$, then
 				$$V'(x) = |f'(x)| \mbox{ for a.e. } x \in [a,b]$$
 				The integral
 				$$\int_a^b |f'| = \int_a^b V' \leq V(b^-) - V(a^+) \leq V[a,b]$$\

 			\item[(ii)].\\
 				If the equality holds in this inequality.\

 				By Theorem 7.24, we have
 				$$\begin{aligned}
 				\int_a^x V'
 				&= \int_a^x |f'|\\
 				&= \int_a^b |f'| - \int_x^b |f'|\\
 				&= V[a,b] - \int_x^b |f'|\\
 				&= V[a,x] + V[x,b] - \int_x^b |f'|\\
 				&\geq V[a,x]
 				\end{aligned}$$
 				for all $x \in [a,b]$.\

 				This completes the prove by Theorem 7.29 and Exercise 7.8.\\

 		\end{enumerate}




 	\item (Exercise 7.10)
 		\begin{enumerate}
 			\item Show that if $f$ is absolutely continuous on $[a,b]$ and $Z$ is a subset of $[a,b]$ of measure zero, then the image set defined by $f(Z) = \{w : w = f(z), z \in Z\}$ also has measure zero. Deduce that the image under $f$ of any measurable subset of $[a, b]$ is measurable. (Compare Theorem 3.33.) (Hint: use the fact that the image of an interval $[a_i, b_i]$ is an interval of length at most $V(b_i) - V (a_i)$.)\\

 			\item Give an example of a strictly increasing Lipschitz continuous function $f$ and a set $Z$ with measure $0$ such that $f^{-1}(Z)$ does not have measure $0$ (and consequently, $f^{-1}$ is not absolutely continuous). (Let $f^{-1}(x) = x + C(x)$ on $[0, 1]$, where $C(x)$ is the Cantor–Lebesgue function.)\
 		\end{enumerate}

 		\textit{\textbf {Proof.}}\\
 		\begin{enumerate}
 			\item.\\
 			Let $\epsilon > 0$, since $f$ is absolutely continuous on $[a, b]$, so is the variation $V$ of $f$ over $[a, b]$, then there exists $\delta > 0$ such that
 				$$\sum_i |V(b_i) - V(a_i)| < \epsilon$$
 			for any nonoverlapping subintervals $[a_i,b_i]$ of $[a,b]$ the sum of whose length $\sum_i (b_i - a_i)$ is less than $\delta$.\

 			Let $Z$ be any subset of $[a,b]$ with measure zero, there exists an open set $G$ contains $Z$ such that $|G| < \delta$.\

 			The open set $G$ can be written as the countable union of nonoverlapping subintervals $[a_i',b_i']$ of $[a,b]$.\

 			Thus
 				$$\sum_i (b_i' - a_i') < \delta$$
 			This implies that
 				$$\begin{aligned}
 				|f(Z)|_e
 				&\leq |f(\cup_i [a_i',b_i'])|_e\\
 				&\leq \sum_i |f([a_i',b_i'])|_e\\
 				&\leq \sum_i [\underset{x \in [a_i',b_i']}{\sup} f(x) - \underset{x \in [a_i',b_i']}{\inf} f(x)]\\
 				&\leq \sum_i [V(b_i') - V(a_i')]\\
 				&< \epsilon
 				\end{aligned}$$

 			So $f(Z)$ is measure zero.\

 			For $E$ be any measurable subset of $[a, b]$, written as $E = F \cup Z$ where $F$ is of type $F_\sigma$, $Z$ is a set with measure zero and $F \cap Z = \phi$.\

 			Note that $F$ is union of compact subsets of $[a, b]$, then $f(F)$ is measurable since $f$ is continuous on $[a, b]$.\

 			Hence $f(E)$ is measurable.\\



 			\item.\\
 			Let $f^{-1} (x) = x + C(x)$ on $[0,1]$.\

 			Since $f^{-1}(x)$ is strictly increasing, its inverse $f(x)$ exists and $f$ is strictly increasing.\

 			Let $x, y \in f^{-1} ([0,1]) = [0,2]$.\

 			Suppose $x < y$ and write $x = p + C(p)$, $y = q + C(q)$ where $p < q$.\

 			Since $f^{-1}(x) = x + C(x) \Rightarrow x = f(x + C(x))$ and the Cantor function $C(x)$ is increasing, then
 				$$\begin{aligned}
 				f(y) - f(x)
 				&= f(q + C(q)) - f(p + C(p))\\
 				&= q - p\\
 				&\leq q + C(q) - p - C(p)\\
 				&= y - x
 				\end{aligned}$$
 			So $f$ is Lipschitz continuous.\

 			Let $Z$ be the Cantor set, which has measure zero.\

 			Since $C(x)$ is constant on each disjoint interval in $[0,1] \setminus Z$, $f^{-1}(x)$ maps each interval to an interval of the same length.\

 			Thus
 				$$|f^{-1}([0,1] \setminus Z)| = |[0,1] \setminus Z| = 1$$
 			Since $f^{-1}([0,1]) = [0,2]$, thus $|f^{-1}(Z)| = 1 > 0$.


 		\end{enumerate}




 \end{enumerate}
\end{document}






