\documentclass[a4paper,11pt]{article}
\usepackage[top=2cm,bottom=2cm,outer=2cm,inner=2cm]{geometry}
\usepackage[utf8]{inputenc}
\usepackage[T1]{fontenc}
\usepackage[inline]{enumitem}
\usepackage{amsfonts}
\usepackage{amsmath}
\usepackage{mathrsfs}
\usepackage{graphicx}


\title{Real Analysis \\ Homework 4}
\author{Yueh-Chou Lee}
\date{\today}
\begin{document}
\maketitle
\begin{enumerate}


% %%%%%%%%%%%%%%%%%%%%%%%%%%%%%%%%%%%%%%%%%%%%%%%%%%%%%%%%%%%%%%%%%%%
% Ex9

	\item (Exercise 10.9)\\
		The \textit{symmetric difference} of two sets $E_1$ and $E_2$ is defined as
			$$E_1 \triangle E_2 = (E_1 - E_2) \cup (E_2 - E_1)$$
		Let $(\mathscr{S}, \Sigma, \mu)$ be a measure space, and identify measurable sets $E_1$ and $E_2$ if $\mu (E_1 \triangle E_2) = 0$. Show that $\Sigma$ is a metric space with distance $d(E_1,E_2) = \mu(E_1 \triangle E_2)$ and if $\mu$ is finite, then $L^p (\mathscr{S}, \Sigma, \mu)$ is separable if and only if $\Sigma$ is $1 \leq p < \infty$.  (For the sufficiency in the second part, Exercise 10.8 may be helpful; for the necessity, let $\{ f_k \}$ be a countable dense set in $L^p (\mathscr{S}, \Sigma, \mu)$ and consider the sets $\{1/2 < f_k \leq 3/2 \}$.)\\
	\newline
	\textit{\textbf {Proof.}}
	\begin{enumerate}
		\item \textit{Show that $\Sigma$ is a metric space with distance $d(E_1,E_2) = \mu(E_1 \triangle E_2)$.}\
		\begin{enumerate}
			\item $\mu \geq 0$ for all measurable set.

			\item $E_1 \triangle E_2 = E_2 \triangle E_1$.

			\item $E_1 \triangle E_2 \subseteq (E_1 \triangle E_3) \cup (E_3 \triangle E_2)$.\
		\end{enumerate}
		By above three, then $\Sigma$ is a metric space with distance $d$.\\

		\item
		$(\Rightarrow)$\\
			If $\mu$ is finite and $L^p (\mathscr{S}, \Sigma, \mu)$ is separable, let $A$ be a countable dense subset of $L^p$.\

			By exercise 10.8, we know that for any $f \in A$, there exists the sequence of simple functions $\{f_k\}$ vanishing outside sets of finite measure such that $f_k \to f$ in $L^p$.\\

			Let $B$ is class of the disjoint union $\underset{i}{\cup} E_i$ satisfies $f_k = \underset{i}{\sum} a_i E_i$ is for any $k$ and any $f \in A$. Then $B \subset \Sigma$ is countable.\\

			For any $E \in \Sigma$, there exists $f_n \in A$ such that $f_n \to \chi_E$ in $L^p$ and exists the sample functions $f_{nm} \to f_n$ in $L^p$.\

			Hence there exists the sequence of simple functions $\{f_j\}$ with $f_j = \underset{l}{\sum} a_{jl} E_{jl}$ where $\cup_l E_{jl} \in B$ such that $f_j \to \chi_E$ in $L^p$.\\

			For any $\epsilon > 0$, there exists $M > 0$ such that for any $j \geq M$, we have $||f_j - \chi_E||_p < \epsilon$.\

			Note that $a_{jl} \to 1$ as $j \to \infty$ for any $l$. Since $\mu$ is finite, then for any $j \geq M$, we have
				$$\begin{aligned}
				\mu^{1/p} (\cup_l E_{jl} \triangle E)
				&= ||\chi_{\cup_l E_{jl}} - \chi_E||_p\\
				&\leq ||\chi_{\cup_l E_{jl}} - \sum a_{jl} E_{jl} ||_p + ||f_j - \chi_E||_p\\
				&\leq ||\chi_{\cup_l E_{jl}} - \sum a_{jl} E_{jl} ||_p + \epsilon
				\end{aligned}$$
			Note that $a_{jl} \to 1$ as $j \to \infty$ for any $l$. Thus $\mu(\cup_l E_{jl} \triangle E) \to 0$ as $j \to \infty$.\\

		$(\Leftarrow)$\\
			Let $B = \{E_k\}$ be a countable dense subset of $\Sigma$.\

			Let $A$ be the set of all linear combinations of characteristic functions of these $E_k$, the coefficients being complex numbers with rational real and imaginary parts.\

			Then $A$ is a countable subset of $L^p (\mathscr{S}, \Sigma, \mu)$.\\

			To see that $A$ is dense, let $f$ be any function in $L^p$, by exercise 10.8, we know that there exists the sequence of simple functions $\{f_k\}$ vanishing outside sets of finite measure such that $f_k \to f$ in $L^p$.\\

			For any $f_k$, let $f_{kl} \to f_k$ with $f_{kl} \in A$ since $B$ is dense, then $||f_{kl} - f_k||_p \to 0$ as $l \to \infty$.\
			Hence there exists $\{f_j\} \subset A$ such that $f_j \to f$ in $L^p$. Thus $A$ is dense in $L^p$.\\
	\end{enumerate}



% %%%%%%%%%%%%%%%%%%%%%%%%%%%%%%%%%%%%%%%%%%%%%%%%%%%%%%%%%%%%%%%%%%%
% Ex10

	\item (Exercise 10.10)\\
		If $\phi$ is a set function whose Jordan decomposition is $\phi = \overline{V} - \underline{V}$, define
			$$\int_E f d\phi = \int_E f d\overline{V} - \int_E f d\underline{V},$$
		provided not both integrals on the right are infinite with the same sign. If $V$ is the total variation of $\phi$ on $E$, and if $|f| \leq M$, prove that $|\int_E f d\phi| \leq MV$.\\
	\newline
	\textit{\textbf {Proof.}}\\
		The functions $\overline{V}$ and $\underline{V}$ are measure since $\overline{V}$, $\underline{V} \geq 0$ and countably additive. We have
			$$\begin{aligned}
			|\int_E f d\phi|
			&= |\int_E f d(\overline{V} - \underline{V})|\\
			&= |\int_E f d\overline{V} - \int_E f d\underline{V}|\\
			&\leq \int_E |f| d\overline{V} + \int_E |f| d\underline{V}\\
			&\leq M \overline{V}(E) + M \underline{V}(E)\\
			&= MV(E)
			\end{aligned}$$



% %%%%%%%%%%%%%%%%%%%%%%%%%%%%%%%%%%%%%%%%%%%%%%%%%%%%%%%%%%%%%%%%%%%
% Ex13

	\item (Exercise 10.13)\\
		Show that the set $P$ of the Hahn decomposition is unique up to null sets. (By a null set for $\phi$, we mean a set $N$ such that $\phi$, we mean a set $N$ such that $\phi(A) = 0$ for every measurable $A \subset N$.)\\
	\newline
	\textit{\textbf {Proof.}}\\
		If $P_1 \cup E - P_1$ and $P_2 \cup E - P_2$ are Hahn decompositions of $E$, then $\phi(A) \geq 0$ if $A \subset P_1$ or $A \subset P_2$, $\phi(A) \leq 0$ if $A \subset E - P_1$ or $A \subset E - P_2$.\

		We said that Hahn decomposition is unique if $\phi(A) = 0$ for any $A \subset P_1 \triangle P_2$, that is
			$$\phi(A) = \phi(P_1 \cap A) + \phi(P_2 \cap A) = 0$$

		Let $N_1 = E - P_1$ and $N_2 = E - P_2$ be two null sets, such that $\phi(N_1) = 0$ and $\phi(N_2) = 0$. Then for any $A \subset P_1 \triangle P_2$, we have
			$$\phi(A) = \phi(P_1 \cap A) + \phi(P_2 \cap A) = \phi((E - P_2) \cap A) + \phi((E - P_1) \cap A) = 0 + 0 = 0,$$
		since $\phi(N_1) = 0$ and $\phi(N_2) = 0$.\

		Assume that $P_1, E - P_1, P_2$ and $E - P_2$ are not in null set. For any $A \subset P_1 \triangle P_2$, then
			$$\phi(A) = \phi(P_1 \cap A) + \phi(P_2 \cap A) = \phi((E - P_2) \cap A) + \phi(P_2 \cap A)$$
		If $\phi(P_1 \cap A) = - \phi(P_2 \cap A)$, then $\phi(A) = 0$.\

		Since $\phi(A) \geq 0$ if $A \subset P_1$ or $A \subset P_2$, we have $\phi(P_1 \cap A) = \phi(P_2 \cap A) = 0$. This is a contradiction since $P_1, P_2$ are not in null set.\

		Hence the set $P$ of the Hahn decomposition is unique up to null sets.\\
		


% %%%%%%%%%%%%%%%%%%%%%%%%%%%%%%%%%%%%%%%%%%%%%%%%%%%%%%%%%%%%%%%%%%%
% Ex15

	\item (Exercise 10.15)\\
		(Converse of H\"{o}lder’s inequality) Let $\mu$ be a $\sigma$-finite measure and $1 \leq p \leq \infty$.
		\begin{enumerate}
			\item
				Show that
					$$||f||_p = \sup |\int fg d\mu|,$$
				where  the supremum is taken over all bounded measurable functions $g$ that vanish outside a set (depending on $g$) of finite measure, and for which $||g||_{p'} \leq 1$ and $\int fg d\mu$ exists. (If $1 < p \leq \infty$ and $||f||_p < \infty$, this can be deduced from Theorem 10.44.)\\

			\item
				Show that a real-valued measurable $f$ belongs to $L^p$ if $fg \in L^1$ for all $g \in L^{p'}$, $\dfrac{1}{p} + \dfrac{1}{p'} = 1$.\\
		\end{enumerate}
	\textit{\textbf {Proof.}}\\
		\begin{enumerate}
			\item For all $||g||_{p'} \leq 1$ then $\int fg d\mu$ exists, so for $1 \leq p \leq \infty, \dfrac{1}{p} + \dfrac{1}{p'} = 1$, we have
				$$|\int fg d\mu|
				\leq \int |fg| d\mu
				\underset{\text{By H\"{o}lder's inequality}}{\leq} ||f||_{p} ||g||_{p'}
				\leq ||f||_p$$
			so $\sup |\int fg d\mu| \leq ||f||_p$.\\

			Since $\mu$ is $\sigma$-finite $\quad \Rightarrow \quad \exists E_j \nearrow \mathscr{S}, \quad \mu(E_j) < \infty$ for all $j$.\

			If $p = 1$, then $p' = \infty$, let
				$$g_j = \left\{ \begin{matrix}
				&1 &\text{, if $x \in E_j$, $f \geq 0$}\\
				&-1 &\text{, if $x \in E_j$, $f < 0$}\\
				&0 &\text{, if $x \notin E_j$}
				\end{matrix} \right.
				\quad \Rightarrow \quad
				0 \leq fg_j \nearrow |f|$$
			By Monotone Convergence Theorem, we have
				$$|\int fg_j d\mu| = \int f g_j \nearrow \int |f|
				\quad \Rightarrow \quad
				\sup |\int fg d\mu| \geq ||f||_1$$
			Thus
				$$||f||_1 = \sup |\int fg d\mu|$$

			If $1 < p \leq \infty$, then $1 \leq p' < \infty$.\

			By Theorem 10.44, we know that for $f \in L^p$, $\forall g \in L^{p'}$ and let $l(g) = \int gf d\mu$, then
				$$||f||_p
				= ||l||
				= \underset{||g||_{p'} \leq 1}{\sup} |l(g)|
				= \underset{||g||_{p'} \leq 1}{\sup} |\int fg d\mu|$$
			But if $||f||_p = + \infty$, let
				$$f_j (x) = \left\{
				\begin{matrix}
				&\min \{ |f|, j\} &\text{if $x \in E_j$}\\
				&0 &\text{if $x \notin E_j$}
				\end{matrix}
				\right.$$
			$$\Rightarrow \quad f_j \in L^p, \quad 0 \leq f_j \nearrow |f|, \quad ||f_j||_p \nearrow ||f||_p$$

			\item
				Suppose that we have a sequence of $L^p$ functions $\{g_k : ||g_k||_{p'} = 1 \}$ where $\int |fg_k| dx > 3^k$\

				Set $g = \sum_{k=1}^\infty 2^{-k} g_k$ so
					$$||g||_{p'} \leq 1 \quad \text{yet} \quad fg \notin L^1$$
				Thus, by Theorem 12.88 Riesz's Theorem, there must be a constant $C$ so that
					$$||fg||_1 \leq C ||g||_{p'}$$

				
		\end{enumerate}


% %%%%%%%%%%%%%%%%%%%%%%%%%%%%%%%%%%%%%%%%%%%%%%%%%%%%%%%%%%%%%%%%%%%
% Ex16

	\item (Exercise 10.16)\\
		Consider a convolution operator $Tf(\mathbf{x}) = \int_{\mathbb{R}^n} f(\mathbf{y}) K(\mathbf{x-y}) d\mathbf{y}$ with $K \geq 0$. If $1 \leq p \leq \infty$ and $||Tf||_p \leq M||f||_p$ for all $f$, show that $||Tf||_{p'} \leq M||f||_{p'}$ for all $f$, $\dfrac{1}{p} + \dfrac{1}{p'} = 1$. (Use Exercise 10.15 to write  $||Tf||_{p'} = \sup_{||g||_p \leq 1} |\int_{\mathbb{R}^n} (Tf) g d\mathbf{x}|$, and note that
			$$\int_{\mathbf{R}^n} (Tf) (\mathbf{x})g(\mathbf{x}) d\mathbf{x} = \int_{\mathbb{R}^n} (T \widetilde{g}) (-\mathbf{y}) f(\mathbf{y}) d\mathbf{y}$$
		where $\widetilde{g}(\mathbf{x}) = g(-\mathbf{x}).$\

		Find a generalization if the hypothesis is instead that $||Tf||_q \leq M ||f||_p$ for all $f$, where $q$ is a fixed index with $1 \leq q \leq \infty$ and $q \neq p$.\\
	\newline
	\textit{\textbf {Proof.}}\\
		By Exercise 10.15, we can write
			$$||Tf||_{p'} = \sup_{||g||_p \leq 1} |\int_{\mathbb{R}^n} (Tf) g d\mathbf{x}|,$$
		so
			$$\begin{aligned}
			||Tf||_{p'}
			&= \sup_{||g||_p \leq 1} |\int_{\mathbb{R}^n} (Tf)(\mathbf{x}) g(\mathbf{x}) d\mathbf{x}|\\
			&= \sup_{||\widetilde{g}||_p \leq 1} |\int_{\mathbb{R}^n} (T\widetilde{g})(\mathbf{-y}) f(\mathbf{y}) d\mathbf{y}|\\
			\text{By H\"{o}lder inequality} \quad
			&\leq \sup_{||\widetilde{g}||_p \leq 1} ||T\widetilde{g}||_{p} ||f||_{p'}\\
			&\leq \sup_{||\widetilde{g}||_p \leq 1} M||\widetilde{g}||_{p} ||f||_{p'}\\
			&\leq M ||f||_{p'}
			\end{aligned}$$

		\textit{\textbf {Generalization:}}\\
		By Exercise 10.15, we can write
			$$||Tf||_{q} = \sup_{||g||_p \leq 1} |\int_{\mathbb{R}^n} (Tf) g d\mathbf{x}|,$$
		so
			$$\begin{aligned}
			||Tf||_{q}
			&= \sup_{||g||_p \leq 1} |\int_{\mathbb{R}^n} (Tf)(\mathbf{x}) g(\mathbf{x}) d\mathbf{x}|\\
			&= \sup_{||\widetilde{g}||_p \leq 1} |\int_{\mathbb{R}^n} (T\widetilde{g})(\mathbf{-y}) f(\mathbf{y}) d\mathbf{y}|\\
			\text{By H\"{o}lder inequality} \quad
			&\leq \sup_{||\widetilde{g}||_p \leq 1} ||T\widetilde{g}||_{q} ||f||_{p}\\
			&\leq \sup_{||\widetilde{g}||_p \leq 1} M||\widetilde{g}||_{p} ||f||_{p}\\
			&\leq M ||f||_{p}
			\end{aligned}$$


\end{enumerate}
\end{document}














