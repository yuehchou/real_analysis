\documentclass[a4paper,11pt]{article}
\usepackage[top=2cm,bottom=2cm,outer=2cm,inner=2cm]{geometry}
\usepackage[utf8]{inputenc}
\usepackage[T1]{fontenc}
\usepackage[inline]{enumitem}
\usepackage{mathrsfs} 
\usepackage{amsfonts}
\usepackage{amsmath}


\title{Real Analysis Homework\\ Chapter 2. Integration Theory\\ Due Date: 12/2}
\author{National Taiwan University, Department of Mathematics\\
R06221012 \hspace{0.2cm} Yueh-Chou Lee}
\date{December 2, 2019}
\begin{document}
\maketitle

%%%%%%%%%%%%%%%%%%%%%%%%%%%%%%%%%%%%%%%%%%%%%%%%%%%%%%%%%%%%%%%%%%%%%%%%%%%

\begin{flushleft}
	\rule[-0.5ex]{17cm}{2pt}\\
		\textbf{Exercise 2.17(a)}\\
	\rule[1.5ex]{17cm}{0.5pt}
		Suppose $f$ is defined on $\mathbb{R}^2$ as follows:
			$$f(x, y) = \left\{\begin{matrix}\begin{aligned}
			&a_n &&\text{if }n \leq x < n + 1\text{ and }n \leq y < n + 1,\, (n \geq 0)\\
			&-a_n &&\text{if }n \leq x < n + 1\text{ and }n + 1 \leq y < n + 2,\,(n \geq 0)\\
			&0 &&\text{o.w.}
			\end{aligned}\end{matrix}\right.$$
		Here $a_n = \sum_{k \leq n} b_k$, with $\{b_k\}$ a positive sequence such that $\sum_{k = 0}^\infty b_k = s < \infty$.\\
		\quad Verify that each slice $f^y$ and $f_x$ is integrable. Also for all $x$, $\int\,f_x(y)\,dy = 0$, and hence $\int(\int\,f(x,y)\,dy)\,dx = 0$.
	\rule[1.0ex]{17cm}{0.5pt}\
\end{flushleft}

\textbf{\textit{Proof.}}

By the definition of $f(x, y)$, we have
	$$f^y(x) =
	\left\{\begin{matrix}\begin{aligned}
	&-a_{\lfloor y \rfloor - 1}
	&&y \geq 1\text{ and }(\lfloor y \rfloor - 1) \leq x < \lfloor y \rfloor\\
	&a_{\lfloor y \rfloor}
	&&y \geq 0\text{ and }\lfloor y \rfloor \leq x < (\lfloor y \rfloor + 1)\\
	&0
	&&\text{o.w.}
	\end{aligned}\end{matrix}\right.$$

Similarly,
	$$f_x(y) =
	\left\{\begin{matrix}\begin{aligned}
	&a_{\lfloor x \rfloor}
	&&x \geq 0\text{ and } \lfloor x \rfloor \leq y < (\lfloor x \rfloor + 1)\\
	&-a_{\lfloor x \rfloor}
	&&x \geq 0\text{ and }(\lfloor x \rfloor + 1) \leq y < (\lfloor x \rfloor + 2)\\
	&0
	&&\text{o.w.}
	\end{aligned}\end{matrix}\right.$$

To show that each slice $f^y(x)$ is integrable, we compute, for $y \geq 1$, then
	$$\begin{aligned}
	\int\,f^y(x)\,dx
	&= \int_{\lfloor y \rfloor - 1}^{\lfloor y \rfloor}\,-a_{\lfloor y \rfloor - 1}\,dx + \int_{\lfloor y \rfloor}^{\lfloor y \rfloor + 1}a_{\lfloor y \rfloor}\,dx\\
	&= a_{\lfloor y \rfloor} - a_{\lfloor y \rfloor - 1}\\
	&= b_{\lfloor y \rfloor} \quad \quad\text{is bounded,}
	\end{aligned}$$

since $\{b_n\}$ is convergent ($\sum_{k = 0}^\infty b_k = s < \infty$). For $0 \leq y < 1$, then
	$$\int\,f^y(x)\,dx = a_0.$$

Similarly, to show that each slice $f_x(y)$ is integrable, we compute
	$$\int\,f_x(y)\,dy
	= \int_{\lfloor x \rfloor}^{\lfloor x \rfloor + 1}\,a_{\lfloor x \rfloor}\,dx + \int_{\lfloor x \rfloor + 1}^{\lfloor x \rfloor + 2}-a_{\lfloor x \rfloor}\,dx
	= a_{\lfloor x \rfloor} - a_{\lfloor x \rfloor}
	= 0$$

So
	$$\int \left(\int\,f(x,y)\,dy\right)\,dx
	= \int \left(\int\,f_x(y)\,dy\right)\,dx
	= \int\,0\,dx
	= 0$$\\\\

%%%%%%%%%%%%%%%%%%%%%%%%%%%%%%%%%%%%%%%%%%%%%%%%%%%%%%%%%%%%%%%%%%%%%%%%%%%

\begin{flushleft}
	\rule[-0.5ex]{17cm}{2pt}\\
		\textbf{Exercise 2.17(b)}\\
	\rule[1.5ex]{17cm}{0.5pt}
		However,
			$$\int\,f^y(x)\,dx
			= \left\{ \begin{matrix} \begin{aligned}
			&a_0 &&\text{if }0 \leq y < 1\\
			&a_n - a_{n - 1} &&\text{if }n \leq y < n + 1\text{ with }n \geq 1
			\end{aligned}\end{matrix}\right.$$
		Hence $y \mapsto \int\,f^y(x)\,dx$ is integrable on $(0,\,\infty)$ and
			$$\int\left(\int\,f(x,\,y)\,dx\right)\,dy = s.$$
	\rule[1.0ex]{17cm}{0.5pt}\
\end{flushleft}

\textbf{\textit{Proof.}}

	$$\begin{aligned}
	\int\left(\int\,f(x,\,y)\,dx\right)\,dy
	&= \sum_{n = 0}^\infty \int_{n}^{n + 1}\left(\int_{\mathbb{R}}\,f^y(x)\,dx\right)\,dy\\
	&= a_0 + \sum_{n = 1}^\infty\,(a_n - a_{n-1})\\\\
	\end{aligned}$$


%%%%%%%%%%%%%%%%%%%%%%%%%%%%%%%%%%%%%%%%%%%%%%%%%%%%%%%%%%%%%%%%%%%%%%%%%%%

\begin{flushleft}
	\rule[-0.5ex]{17cm}{2pt}\\
		\textbf{Exercise 2.17(c)}\\
	\rule[1.5ex]{17cm}{0.5pt}
		Note that $\int_{\mathbb{R} \times \mathbb{R}} |f(x,y)|\,dxdy = \infty$.
	\rule[1.0ex]{17cm}{0.5pt}\
\end{flushleft}

\textbf{\textit{Proof.}}

Since $0 \leq |f(x,y)| < \infty$, by Tonelli's theorem, we then have
	$$\begin{aligned}
	\int_{\mathbb{R} \times \mathbb{R}} |f(x,y)|\,dxdy
	&= \int_{\mathbb{R}} \int_\mathbb{R} |f_x(y)|\,dydx\\
	&= \sum_{n = 0}^{\infty}\int_{n}^{n+1}\int_\mathbb{R} |f_x(y)|\,dydx\\
	&= \sum_{n = 0}^{\infty}2a_n
	\end{aligned}$$

However, each of the $a_n \geq a_0 > 0$, then $\sum_{n = 0}^{\infty}2a_n$ is divergent, so
	$$\int_{\mathbb{R} \times \mathbb{R}} |f(x,y)|\,dxdy = \infty$$\\\\



%%%%%%%%%%%%%%%%%%%%%%%%%%%%%%%%%%%%%%%%%%%%%%%%%%%%%%%%%%%%%%%%%%%%%%%%%%%

\begin{flushleft}
	\rule[-0.5ex]{17cm}{2pt}\\
		\textbf{Exercise 2.18}\\
	\rule[1.5ex]{17cm}{0.5pt}
		Let $f$ be a measurable finite-valued function on $[0,\,1]$, and suppose that $|f(x) - f(y)|$ is integrable on $[0,\,1] \times [0,\,1]$. Show that $f(x)$ is integrable on $[0,\,1]$.
	\rule[1.0ex]{17cm}{0.5pt}\
\end{flushleft}

\textbf{\textit{Proof.}}

Let the function $g(x,y) = |f(x) - f(y)|$. By Fubini's theorem, since $g$ is integrable on $[0,\,1] \times [0,\,1]$, the slice $g^y(x)$ is integrable for a.e. $y \in [0,\,1]$. Fixed $y \in [0,\,1]$, then
	$$f(x) - f(y) \leq |f(x) - f(y)|$$

Also, we can use the monotonicity of the integral to see that
	$$\int_{[0,1]} f(x) - f(y)\,dx
	\leq \int_{[0,1]} |f(x) - f(y)|\,dx
	\leq \infty$$

That is
	$$\begin{aligned}
	\int_{[0,1]} f(x)\,dx
	&\leq \int_{[0,1]} f(y)\,dx + \int_{[0,1]} |f(x) - f(y)|\,dx\\
	&= f(y) +  \int_{[0,1]} |f(x) - f(y)|\,dx
	\end{aligned}$$

Since $f(y)$ and $\int_{[0,1]} |f(x) - f(y)|\,dx$ are finite, then $f(x)$ is integrable on $[0,\,1]$.\\\\


%%%%%%%%%%%%%%%%%%%%%%%%%%%%%%%%%%%%%%%%%%%%%%%%%%%%%%%%%%%%%%%%%%%%%%%%%%%

\begin{flushleft}
	\rule[-0.5ex]{17cm}{2pt}\\
		\textbf{Exercise 2.19}\\
	\rule[1.5ex]{17cm}{0.5pt}
		Suppose $f$ is integrable on $\mathbb{R}^d$. For each $\alpha > 0$, let $E_\alpha = \{x\,:\,|f(x)| > \alpha\}$. Prove that
			$$\int_{\mathbb{R}^d} |f(x)|\,dx
			= \int_0^\infty m(E_\alpha)\,d\alpha$$
	\rule[1.0ex]{17cm}{0.5pt}\
\end{flushleft}

\textbf{\textit{Proof.}}

Observe that
	$$m(E_\alpha) = \int_{\mathbb{R}^d} \chi_{E_\alpha}(x)\,dx.$$

Let $f(\alpha,\,x) = \chi_{E_\alpha}(x)$. By Tonelli's theorem, since $f$ is non-negative measurable function on $(0,\,\infty) \times \mathbb{R}^d$, then
	$$\begin{aligned}
	\int_0^\infty\,m(E_\alpha)\,d\alpha
	&= \int_0^\infty \int_{\mathbb{R}^d}\,f(\alpha,\,x)\,dxd\alpha\\
	&= \int_{\mathbb{R}^d} \left( \int_0^\infty\,\chi_{E_\alpha}(x)\,d\alpha\right)\,dx\\
	&= \int_{\mathbb{R}^d}\,|f(x)|\,dx\\\\\
	\end{aligned}$$



%%%%%%%%%%%%%%%%%%%%%%%%%%%%%%%%%%%%%%%%%%%%%%%%%%%%%%%%%%%%%%%%%%%%%%%%%%%
\newpage

\begin{flushleft}
	\rule[-0.5ex]{17cm}{2pt}\\
		\textbf{Exercise 2.20}\\
	\rule[1.5ex]{17cm}{0.5pt}
		The problem (highlighted in the discussion preceding Fubini’s theorem) that certain slices of measurable sets can be non-measurable may be avoided by restricting attention to Borel measurable functions and Borel sets. In fact, prove the following:

		\quad Suppose $E$ is a Borel set in $\mathbb{R}^2$. Then for every $y$, the slice $E^y$ is a Borel set in $\mathbb{R}$.

		[Hint: Consider the collection $\mathcal{C}$ of subsets $E$ of $\mathbb{R}^2$ with the property that each slice $E^y$ is a Borel set in $\mathbb{R}$. Verify that $\mathcal{C}$ is a $\sigma$-algebra that contains the open sets.]
	\rule[1.0ex]{17cm}{0.5pt}\
\end{flushleft}

\textbf{\textit{Proof.}}

Define
	$$\mathcal{C}
	= \{E \subset \mathbb{R}^2\,|\,\forall y,\,E^y\ \text{is Borel.}\}$$

We first prove the following.

	\begin{enumerate}
		\item Prove that $\mathcal{C}$ is $\sigma$-algebra.\\

		\textbf{\textit{Proof.}}

		It's clear that $\mathcal{C}$ is non-empty.\\

		Next, we will show that $E \in \mathcal{C}$, then $E^c \in \mathcal{C}$.

		Let $E \in \mathcal{C}$. Since for any $y$ the slice $E^y$ is Borel and $(E^c)^y = (E^y)^c$ is Borel (the Borel sets are a $\sigma$-algebra), then $E^c \in \mathcal{C}$.\\

		Finally, suppose that $\{E_k\}_{k = 1}^\infty$ is a countable collection of sets in $\mathcal{C}$. Since
			$$\left(\bigcup_{k = 1}^\infty\,E_k\right)^y
			= \bigcup_{k = 1}^\infty\,E_k^y,$$

		and the fact that each of the $E_k^y$ is Borel, then $\bigcup_{k = 1}^\infty\,E_k^y$ must be Borel so as $\left(\bigcup_{k = 1}^\infty\,E_k\right)^y$.\\

		By the above, we then know that $\mathcal{C}$ is $\sigma$-algebra.\\

		\item Prove that if $E \subset \mathbb{R}^2$ is open then $E \in \mathcal{C}$.\\

		\textbf{\textit{Proof.}}\\

		If $E$ is an open cube, then we must have $E = (a,\,b) \times (a+h,\,b+h)$ for some $h > 0$.

		Take the slice $E^y$ which is the same as $(a,\,b) + y$ and $y \in \mathbb{R}$. In geometric, we simply have the line segment $(a,\,b)$ translated in the $y$-direction a distance $y$.

		So $E^y$ is an open interval and hence is also a Borel set.\\

		If $E$ is a closed cube, then the slice $E^y$ is a closed interval and can be written as the countable intersection of intervals
			$$\bigcap_n\,\left(a - \frac{1}{n},\,b + \frac{1}{n}\right)$$

		Thus, $E^y$ is a Borel set.\\

		Then, we use the fact that every open set in $\mathbb{R}^2$ can be written as the countable union of almost disjoint closed cubes such as
			$$E = \bigcup_{j = 1}^\infty\,Q_k$$

		where each of the $Q_k$ is closed cube. Since $Q_k \in \mathcal{C}$ for all $k$, then $E \in \mathcal{C}$.\\
	\end{enumerate}

	In the end, recall that the Borel sets is the smallest $\sigma$-algebra containing the open sets. Also, we have shown that $\mathcal{C}$ contains the open sets. Therefore, $\mathcal{C}$ contains the Borels sets.\\\\


%%%%%%%%%%%%%%%%%%%%%%%%%%%%%%%%%%%%%%%%%%%%%%%%%%%%%%%%%%%%%%%%%%%%%%%%%%%

\begin{flushleft}
	\rule[-0.5ex]{17cm}{2pt}\\
		\textbf{Exercise 2.21(a)}\\
	\rule[1.5ex]{17cm}{0.5pt}
		Suppose that $f$ and $g$ are measurable functions on $\mathbb{R}^d$. Prove that $f(x - y)g(y)$ is measurable on $\mathbb{R}^{2d}$.
	\rule[1.0ex]{17cm}{0.5pt}\
\end{flushleft}

\textbf{\textit{Proof.}}

Since $f$ is a measurable function on $\mathbb{R}^d$, then by \textbf{Proposition 3.9} in the textbook, we have	$\tilde{f}(x, y) = f(x - y)$ is measurable on $\mathbb{R}^d \times \mathbb{R}^d$.\\

Also, since $g$ is a measurable function on $\mathbb{R}^d$, then by \textbf{Corollary 3.7} in the textbook, we have $\tilde{g}(x, y) = g(y)$ is measurable on $\mathbb{R}^d \times \mathbb{R}^d$.\\

By above two, we know that $f(x - y)g(y)$ is measurable on $\mathbb{R}^d \times \mathbb{R}^d = \mathbb{R}^{2d}$.\\\\





%%%%%%%%%%%%%%%%%%%%%%%%%%%%%%%%%%%%%%%%%%%%%%%%%%%%%%%%%%%%%%%%%%%%%%%%%%%

\begin{flushleft}
	\rule[-0.5ex]{17cm}{2pt}\\
		\textbf{Exercise 2.21(b)}\\
	\rule[1.5ex]{17cm}{0.5pt}
		Suppose that $f$ and $g$ are measurable functions on $\mathbb{R}^d$. Show that if $f$ and $g$ are integrable on $\mathbb{R}^d$, then $f(x - y)g(y)$ is integrable on $\mathbb{R}^{2d}$.
	\rule[1.0ex]{17cm}{0.5pt}\
\end{flushleft}

\textbf{\textit{Proof.}}

Since $f$ and $g$ are measurable functions on $\mathbb{R}^d$, then by \textbf{Tonelli's theorem}, we know that
	$$\int_{\mathbb{R}^{2d}}\,|f(x - y)g(y)|\,d(x,y)
	= \int_{\mathbb{R}^{d}} \int_{\mathbb{R}^{d}}\,|f(x - y)g(y)|\,dxdy$$

Since $f$ and $g$ are integrable on $\mathbb{R}^d$, then $\int_{\mathbb{R}^{d}}\,f = M < \infty$ and $\int_{\mathbb{R}^{d}}\,g = N < \infty$. Thus,
	$$\begin{aligned}
	\int_{\mathbb{R}^{d}} \int_{\mathbb{R}^{d}}\,|f(x - y)g(y)|\,dxdy
	&= \int_{\mathbb{R}^{d}} \int_{\mathbb{R}^{d}}\,|f(x - y)||g(y)|\,dxdy\\
	&= \int_{\mathbb{R}^{d}}\,|g(y)|\int_{\mathbb{R}^{d}}\,|f(x - y)|\,dxdy\\
	&= \int_{\mathbb{R}^{d}}\,|g(y)|\,M\,dy\\
	&= MN < \infty
	\end{aligned}$$

So $f(x - y)g(y)$ is integrable on $\mathbb{R}^{2d}$.\\\\


%%%%%%%%%%%%%%%%%%%%%%%%%%%%%%%%%%%%%%%%%%%%%%%%%%%%%%%%%%%%%%%%%%%%%%%%%%%

\begin{flushleft}
	\rule[-0.5ex]{17cm}{2pt}\\
		\textbf{Exercise 2.21(c)}\\
	\rule[1.5ex]{17cm}{0.5pt}
		Recall the definition of convolution of $f$ and $g$ given by
			$$(f \ast g)
			= \int_{\mathbb{R}^d} f(x - y)g(y)\,dy$$
		Suppose that $f$ and $g$ are measurable functions on $\mathbb{R}^d$. Show that $f \ast g$ is well defined for a.e. $x$ (that is, $f(x - y)g(y)$ is integrable on $\mathbb{R}^d$ for a.e. $x$).
	\rule[1.0ex]{17cm}{0.5pt}\
\end{flushleft}

\textbf{\textit{Proof.}}

Since we know that $f(x - y)g(y)$ is integrable on $\mathbb{R}^{2d}$ by \textbf{Exercise 2.21(b)}, then by \textbf{Fubini's theorem}, so for a.e. $x \in \mathbb{R}^d$ such that
	$$\int_{\mathbb{R}^d}\,f(x - y)g(y)\,dy = (f \ast g)(x) < \infty$$

Hence $f \ast g$ is well defined for a.e. $x$.\\\\


%%%%%%%%%%%%%%%%%%%%%%%%%%%%%%%%%%%%%%%%%%%%%%%%%%%%%%%%%%%%%%%%%%%%%%%%%%%

\begin{flushleft}
	\rule[-0.5ex]{17cm}{2pt}\\
		\textbf{Exercise 2.21(d)}\\
	\rule[1.5ex]{17cm}{0.5pt}
		Suppose that $f$ and $g$ are measurable functions on $\mathbb{R}^d$. Show that $f \ast g$ is integrable whenever $f$ and $g$ are integrable, and that
			$$||f \ast g ||_{L^1(\mathbb{R}^d)}
			\leq ||f||_{L^1(\mathbb{R}^d)} ||g||_{L^1(\mathbb{R}^d)},$$
		with equality if $f$ and $g$ are non-negative.
	\rule[1.0ex]{17cm}{0.5pt}\
\end{flushleft}

\textbf{\textit{Proof.}}

From \textbf{Exercise 2.21(b)} and \textbf{Exercise 2.21(c)}, we have
	$$\begin{aligned}
	||f \ast g ||_{L^1(\mathbb{R}^d)}
	&= \int_{\mathbb{R}^d}\,|(f \ast g)(x)|\,dx\\
	&= \int_{\mathbb{R}^d}\left|\int_{\mathbb{R}^d} f(x - y)g(y)\,dy\right|\,dx\\
	&\leq \int_{\mathbb{R}^d}\int_{\mathbb{R}^d}\,|f(x - y)g(y)|\,dydx\\
	&= MN = ||f||_{L^1(\mathbb{R}^d)} ||g||_{L^1(\mathbb{R}^d)}
	\end{aligned}$$

Moreover, if $f$ and $g$ are positive functions, then $|f(x - y)g(y)| = f(x - y)g(y)$, so the equality holds.\\\\



%%%%%%%%%%%%%%%%%%%%%%%%%%%%%%%%%%%%%%%%%%%%%%%%%%%%%%%%%%%%%%%%%%%%%%%%%%%
\newpage
\begin{flushleft}
	\rule[-0.5ex]{17cm}{2pt}\\
		\textbf{Exercise 2.24(a)}\\
	\rule[1.5ex]{17cm}{0.5pt}
		Consider the convolution
			$$(f \ast g)(x) = \int_{\mathbb{R}^d} f(x - y)g(y)\,dy.$$
		Show that $f \ast g$ is uniformly continuous when $f$ is integrable and $g$ bounded.
	\rule[1.0ex]{17cm}{0.5pt}\
\end{flushleft}

\textbf{\textit{Proof.}}

If $f$ is integrable and $\exists M \geq 0$ such that $|g| \leq M$. So $\forall x,\,z \in \mathbb{R}^d$, we have
	$$\begin{aligned}
	|(f \ast g)(x) - (f \ast g)(z)|
	&= \left|\int_{\mathbb{R}^d}\,(f(x - y) - f(z - y))f(y)\,dy\right|\\
	&\leq M \int_{\mathbb{R}^d}\,|f(x - y) - f(z - y)|\,dy\\
	&= M \int_{\mathbb{R}^d}\,|f(- y) - f(z - y)|\,dy.
	\end{aligned}$$

Since $f \in L^1(\mathbb{R}^d)$, by \textbf{Proposition 2.5} in the textbook, $\forall \varepsilon > 0$, $\exists \delta$ such that $||z - x|| < \delta\ \Rightarrow\ ||f(y) - f(y - (z - x))||_{L^1} < \varepsilon$. Thus,
	$$|(f \ast g)(x) - (f \ast g)(z)| \leq M ||f(y) - f(y - (z - x))||_{L^1} < M\varepsilon.$$

Hence, the convolution $(f \ast g)(x)$ is uniformly continuous.\\\\\\


%%%%%%%%%%%%%%%%%%%%%%%%%%%%%%%%%%%%%%%%%%%%%%%%%%%%%%%%%%%%%%%%%%%%%%%%%%%

\begin{flushleft}
	\rule[-0.5ex]{17cm}{2pt}\\
		\textbf{Exercise 2.24(b)}\\
	\rule[1.5ex]{17cm}{0.5pt}
		If in addition g is integrable, prove that $(f \ast g)(x) \to 0$ as $|x| \to \infty$.
	\rule[1.0ex]{17cm}{0.5pt}\
\end{flushleft}

\textbf{\textit{Proof.}}

By \textbf{Exercise 2.21(d)}, if $f,\,g \in L^1(\mathbb{R}^d)$, then $(f \ast g)(x)$ will be $L^1(\mathbb{R}^d)$. Moreover, by \textbf{Exercise 2.24(a)}, we have $(f \ast g)(x)$ is uniformly continuous, and integrable. By \textbf{Exercise 2.6(b)}, then
	$$\underset{|x| \to \infty}{\lim}\,(f \ast g)(x) = 0.$$


%%%%%%%%%%%%%%%%%%%%%%%%%%%%%%%%%%%%%%%%%%%%%%%%%%%%%%%%%%%%%%%%%%%%%%%%%%%
	% \begin{flushleft}
	% 	\rule[-0.5ex]{17cm}{2pt}\\
	% 		\textbf{Theorem}\\
	% 	\rule[1.5ex]{17cm}{0.5pt}
			
	% 	\rule[1.0ex]{17cm}{0.5pt}\
	% \end{flushleft}
	% \textbf{\textit{Proof.}}\\

\end{document}






