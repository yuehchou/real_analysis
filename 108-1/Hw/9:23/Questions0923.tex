\documentclass[a4paper,11pt]{article}
\usepackage[top=2cm,bottom=2cm,outer=2cm,inner=2cm]{geometry}
\usepackage[utf8]{inputenc}
\usepackage[T1]{fontenc}
\usepackage[inline]{enumitem}
\usepackage{mathrsfs} 
\usepackage{amsfonts}
\usepackage{amsmath}


\title{Real Analysis Homework 1\\ Due Date: 9/23}
\begin{document}
\maketitle

1. Please prove the uniqueness of Theorem 1.3 in the textbook.\\\\


2. Let $R$ be a rectangle in $\mathbb{R}^n$. Prove $\sigma(R) = |R| = m_*(R)$.\\\\


3. \textit{Theorem 1.3} states that every open set in $\mathbb{R}$ is the disjoint union of open intervals. The analogue in $\mathbb{R}^d$, $d \geq 2$, is generally false. Prove the following:

\begin{enumerate}
	\item [(a)] An open disc in $\mathbb{R}^2$ is not the disjoint union of open rectangles.\\
	{[}
	Hint: What happens to the boundary of any of these rectangles?
	{]}

	\item [(b)] An open connected set $\Omega$ is the disjoint union of open rectangles if and only if $\Omega$ is itself an open rectangle.\\
\end{enumerate}



4. At the start of the theory, one might define the outer measure by taking coverings by rectangles instead of cubes. More precisely, we define
	$$m_*^\mathcal{R}(E) = \inf \sum_{j = 1}^\infty |R_j|,$$
where the inf is now taken over all countable coverings $E \subset \cup_{j = 1}^\infty \mathcal{R}_j$ by (closed) rectangles.

Show that this approach gives rise to the same theory of measure developed in the text, by proving that $m_*(E) = m_*^\mathcal{R}(E)$ for every subset $E$ of $\mathbb{R}^d$.\\
{[}
Hint: Use \textit{Lemma 1.1}.
{]}\\


\end{document}