\documentclass[a4paper,11pt]{article}
\usepackage[top=2cm,bottom=2cm,outer=2cm,inner=2cm]{geometry}
\usepackage[utf8]{inputenc}
\usepackage[T1]{fontenc}
\usepackage[inline]{enumitem}
\usepackage{mathrsfs} 
\usepackage{amsfonts}
\usepackage{amsmath}


\title{Real Analysis Homework\\ Chapter 1. Measure theory\\ Due Date: 9/23}
\author{National Taiwan University, Department of Mathematics\\
R06221012 \hspace{0.2cm} Yueh-Chou Lee}
\date{\today}
\begin{document}
\maketitle

\begin{flushleft}
	\rule[-0.5ex]{17cm}{2pt}\\
		\textbf{Recall Theorem 1.3:}\\
	\rule[1.5ex]{17cm}{0.5pt}
		Every open subset $\mathcal{O}$ of $\mathbb{R}$ can be writen uniquely as a countable union of disjoint open intervals.
	\rule[1.0ex]{17cm}{0.5pt}\
\end{flushleft}

\textbf{Exercise of 9/11:}

1. Please prove the uniqueness of Theorem 1.3.\\

\textbf{\textit{Proof.}}

Suppose that open subset $\mathcal{O}$ of $\mathbb{R}$ can be writen in two different countable union of disjoint open intervals $\mathcal{I}$ and $\mathcal{\widetilde{I}}$.\\

Fixed $x \in \mathcal{O}$, and
	$$\mathcal{I}_x
	= \underset{y \in Y}{\cup} (\mathcal{I}_x
	\cap \mathcal{\widetilde{I}}_y).$$

Then $\underset{y \in Y}{\cup} (\mathcal{I}_x \cap \mathcal{\widetilde{I}}_y)$ are disjoint open subsets of $\mathcal{I}_x$.

But $\mathcal{I}_x$ is connected. So $\mathcal{I}_x \cap \mathcal{\widetilde{I}}_y \neq \emptyset$, and thus $\mathcal{I}_x \cap \mathcal{\widetilde{I}}_y = \mathcal{I}_x$ for some $\mathcal{\widetilde{I}}_y$, that is $\mathcal{I}_x \subseteq \mathcal{\widetilde{I}}_y$.\\

Then prove similarly that for each $y \in Y$, there must be exactly one $x$ so that $\mathcal{\widetilde{I}}_y \subseteq \mathcal{I}_x$ and $\mathcal{\widetilde{I}}_y \cap \mathcal{I}_{x'} = \emptyset$ if $x' \neq x$.\\\\


\begin{flushleft}
	\rule[-0.5ex]{17cm}{2pt}\\
		\textbf{Recall Definition:}\\
	\rule[1.5ex]{17cm}{0.5pt}
		Let $E \in \mathbb{R}^n$ and $\{Q_k\}$ be the covering of $E$ and $Q_k$ be the closed cubes, then we define that
		$$\begin{aligned}
		\text{Jordan outer meaure: }
		&\sigma(E)
		= \underset{N \in \mathbb{N}}{\inf}\,\Sigma_{k=1}^N\,|Q_k|\\
		&\text{and}\\
		\text{Lebesgue outer meaure: }
		&m_*(E) = \underset{\{Q_k\} \text{: countable set}}{\inf} \Sigma_{k=1}^{\infty}\,|Q_k|
		\end{aligned}$$
	\rule[1.0ex]{17cm}{0.5pt}\
\end{flushleft}


\textbf{Exercise of 9/16:}

2. Let $R$ be a rectangle in $\mathbb{R}^n$. Prove $\sigma(R) = |R| = m_*(R)$.\\

\textbf{\textit{Proof.}}

(1) Prove $\sigma(R) = |R|$.\\

By Example 2 in the text book (p.11), we know that $|R| \leq \sigma(R)$.\\

To obtain the reverse inequality, consider a grid in $\mathbb{R}^d$ formed by cubes of side length $1/k$.

Then, if $\mathcal{Q}$ consists of the (finite) collection of all cubes entirely contained in $R$, and $\mathcal{Q}'$ is the (finite) collection of all cubes that intersect $R$ and the complement of $R$.\\

Note that $R \subset \overset{N}{\underset{i=1,\,Q \in (\mathcal{Q} \cup \mathcal{Q}')}{\cup}} Q_i$. Also, a simple argument yields
	$$\overset{N}{\underset{i=1}{\sum}}\,|Q_i| \leq |R|.$$

Moreover, there are $O(k^{d-1})$ cubes in $\mathcal{Q}'$  and these cubes have volume $k^{-d}$, so that

$\overset{N}{\underset{i=1}{\sum}}|Q| = O(1/k)$. Hence
	$$\overset{N}{\underset{i=1}{\sum}} |Q|
	\leq |R| + O(1/k),$$

and letting k tend to infinity yields $\sigma(R)\,\leq\,|R|$.

Therefore, $\sigma(R) = |R|$.\\\\


(2) Prove $m_*(R) = |R|$.\\

By Example 2 in the text book (p.11), we know that $|R| \leq m_*(R)$.\\

To obtain the reverse inequality, consider a grid in $\mathbb{R}^d$ formed by cubes of side length $1/k$.

Then, if $\mathcal{Q}$ consists of the (finite) collection of all cubes entirely contained in $R$, and $\mathcal{Q}'$ is the (finite) collection of all cubes that intersect $R$ and the complement of $R$.\\

Note that $R \subset \cup_{Q \in (\mathcal{Q} \cup \mathcal{Q}')} Q$. Also, a simple argument yields
	$$\underset{q \in \mathcal{Q}}{\sum}\,|Q| \leq |R|.$$

Moreover, there are $O(k^{d-1})$ cubes in $\mathcal{Q}'$  and these cubes have volume $k^{-d}$, so that

$\underset{Q \in \mathcal{Q}'}{\sum}|Q| = O(1/k)$. Hence
	$$\underset{Q \in (\mathcal{Q} \cup \mathcal{Q}')}{\sum} |Q|
	\leq |R| + O(1/k),$$

and letting k tend to infinity yields $m_*(R)\,\leq\,|R|$.

Therefore, $m_*(R) = |R|$.\\

By (1) and (2), then $\sigma(R) = |R| = m_*(R)$.\\\\


\textbf{Exercise of 9/18:}

3. \textit{Theorem 1.3} states that every open set in $\mathbb{R}$ is the disjoint union of open intervals. The analogue in $\mathbb{R}^d$, $d \geq 2$, is generally false. Prove the following:

\begin{enumerate}
	\item [(a)] An open disc in $\mathbb{R}^2$ is not the disjoint union of open rectangles.\\
	{[}
	Hint: What happens to the boundary of any of these rectangles?
	{]}

	\item [(b)] An open connected set $\Omega$ is the disjoint union of open rectangles if and only if $\Omega$ is itself an open rectangle.
\end{enumerate}

\textbf{\textit{Proof.}}
\begin{enumerate}
	\item [(a)]
		Suppose for a contradiction that an open disc $\mathcal{O} \subset \mathbb{R}^2$ and $\mathcal{O} = \underset{n \in \mathbb{N}}{\cup}R_n$ where $R_n$ are the open rectangles and $R_i \cap R_j = \emptyset$ for $i \neq j$.\\

		Choose some open rectangle $R_1 \in \mathcal{O}$ and let $x \in \partial R_1$. Then, for all $\epsilon > 0$, $B(x,\epsilon) \cap R_1 \neq \emptyset$ and $B(x,\epsilon) \cap R_1^c \neq \emptyset$. Hence, $x \notin R_1$, and so there must be an open rectangle $R_2 \in \mathcal{O}$ with $x \in R_2$. This implies that there is $\epsilon_0 > 0$ such that $B(x,\epsilon_0) \subset R_2$.\\

		By our previous observation, $B(x,\epsilon_0) \cap R_1 \neq \emptyset$. Taken together, then $R_1 \cap R_2 \neq \emptyset$, which is a contradiction with the fact that $\mathcal{O}$ is the disjoint union of open rectangles.\\

	\item [(b)]
		($\Rightarrow$)
		Let $\Omega$ be the disjoint union of open rectangles. Suppose, to the contrary, that $\Omega$ is not itself an open rectangle.\\

		Then, $\Omega$ contains at least two open rectangles. By the argument in part (a), so these rectangles cannot be disjoint, which is a contradiction. Hence, it must be that $\Omega$ is itself an open rectangle.\\

		($\Leftarrow$) Let $\Omega$ be an open rectangle. Then $\Omega$ is the disjoint union of a single open rectangle.\\\\
\end{enumerate}


4. At the start of the theory, one might define the outer measure by taking coverings by rectangles instead of cubes. More precisely, we define
	$$m_*^\mathcal{R}(E) = \inf \sum_{j = 1}^\infty |R_j|,$$
where the inf is now taken over all countable coverings $E \subset \cup_{j = 1}^\infty \mathcal{R}_j$ by (closed) rectangles.

Show that this approach gives rise to the same theory of measure developed in the text, by proving that $m_*(E) = m_*^\mathcal{R}(E)$ for every subset $E$ of $\mathbb{R}^d$.\\
{[}
Hint: Use \textit{Lemma 1.1}.
{]}\\

\textbf{\textit{Proof.}}
\begin{enumerate}
	\item [(1)]
		Let $E$ be some subset of $\mathbb{R}^d$.

		First, we begin by considering the outer measure of $E$, given by $m_*(E) = \inf \sum_{j=1}^\infty |Q_j|$. Consider any covering of $E$ by closed cubes $E \subset \cup_{j=1}^\infty Q_j$.\\

		For each $Q_j$ with side length $l_j$, we can pick the rectangle $R_j$, whose sides have length\\$r_1 = l + \dfrac{\epsilon}{j}\,l^{d-1}$ and $r_k = l$ for every other side. Then $R_j \supset E_j$ and
			$$|R_j| = |Q_j| + \frac{\epsilon}{2^j}.$$
		So
			$$\sum_{j=1}^\infty |Q_j| < \sum_{j=1}^\infty |R_j| = \sum_{j=1}^\infty |Q_j| + \epsilon.$$
		Passing to the infimum gives
			$$m_*^R(E) \leq \sum_{j=1}^\infty |R_j| \leq m_*(E) + \epsilon.$$
		Then letting $\epsilon \to 0$ gives $m_*^R(E) \leq m_*(E)$.\\

	\item [(2)]
		Now begin with a covering of $E$ by rectangles $R_j$. Then we divide $\mathbb{R}^n$ into a grid of cubes with side length $\frac{1}{k}$. Then we can define $Q_{j,k}$ to be the set of rectangles that have non-empty intersection with $R_j$.\\

		We need to be precise about how much extra measure these cubes add. We know that there are $Ck^{d-1}$ that could intersect partially intersect $R_j$ for some suitably large $C$. The total volume these add is $\frac{C}{k}$ and so if $k \geq 2^j C$, then the total volume is less than $\frac{\epsilon}{2^j}$.

		That is
			$$\underset{k}{\sum} |Q_{j,k}| \leq |R_j| + \frac{\epsilon}{2^j}.$$
		Then
			$$\underset{j}{\sum} |R_j| \leq \underset{j}{\sum} \underset{k}{\sum} |Q_{j,k}| \leq \underset{j}{\sum} |R_j| + \epsilon.$$
		Letting $\epsilon \to 0$ again gives $m_*(E) \leq m_*^R(E)$.
\end{enumerate}
	By (1) and (2), then $m_*(E) = m_*^\mathcal{R}(E)$.


% 5. If a set $E$ is the countable union of almost disjoint cubes $E = \cup_{j=1}^\infty Q_j$. Show that the following equation holds.
	% $$m_*(E) = \sum_{}$$

%%%%%%%%%%%%%%%%%%%%%%%%%%%%%%%%%%%%%%%%%%%%%%%%%%%%%%%%%%%%%%%%%%%%%%%%%%%
	% \begin{flushleft}
	% 	\rule[-0.5ex]{17cm}{2pt}\\
	% 		\textbf{Theorem}\\
	% 	\rule[1.5ex]{17cm}{0.5pt}
			
	% 	\rule[1.0ex]{17cm}{0.5pt}\
	% \end{flushleft}
	% \textbf{\textit{Proof.}}\\

\end{document}