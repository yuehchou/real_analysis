\documentclass[a4paper,11pt]{article}
\usepackage[top=2cm,bottom=2cm,outer=2cm,inner=2cm]{geometry}
\usepackage[utf8]{inputenc}
\usepackage[T1]{fontenc}
\usepackage[inline]{enumitem}
\usepackage{mathrsfs} 
\usepackage{amsfonts}
\usepackage{amsmath}


\title{Real Analysis Homework\\ Chapter 4. Product Measures\\ Due Date: 12/30}
\author{National Taiwan University, Department of Mathematics\\
R06221012 \hspace{0.2cm} Yueh-Chou Lee}
\date{December 30, 2019}
\begin{document}
\maketitle

%%%%%%%%%%%%%%%%%%%%%%%%%%%%%%%%%%%%%%%%%%%%%%%%%%%%%%%%%%%%%%%%%%%%%%%%%%%

\begin{flushleft}
	\rule[-0.5ex]{17cm}{2pt}\\
		\textbf{Exercise 1}\\
	\rule[1.5ex]{17cm}{0.5pt}
		Let $A \subseteq X$ and let $B$ be a $\nu$-measurable subset of $Y$. If $A \times B$ is measurable with respect to the product measure $\mu \times \nu$, is $A$ necessarily measurable with respect to $\mu$?
	\rule[1.0ex]{17cm}{0.5pt}\
\end{flushleft}

\textbf{\textit{Proof.}}

No, $A$ is NOT necessarily measurable with respect to $\mu$.\\

If $X = Y = \text{Borel $\sigma$ algebra of } \mathbb{R}$, $\mu = \nu = \text{Lebesgue measue}$ and $A$ is a Lebesgue measuable set in $\mathbb{R}$ which is not Borel set, then $A \times \mathbb{R}$ is measurable with respect to the product measure but $A \notin X$, that is $A$ may not be measurable with respect to $\mu$.


%%%%%%%%%%%%%%%%%%%%%%%%%%%%%%%%%%%%%%%%%%%%%%%%%%%%%%%%%%%%%%%%%%%%%%%%%%%

\begin{flushleft}
	\rule[-0.5ex]{17cm}{2pt}\\
		\textbf{Exercise 2}\\
	\rule[1.5ex]{17cm}{0.5pt}
		Let $\mathbb{N}$ be the set of natural numbers, $\mathcal{M} = 2^{\mathbb{N}}$, and $c$ the counting measure defined by setting $c(E)$ equal to the number of points in $E$ if $E$ is finite and $\infty$ if $E$ is an infinite set. Prove that every function $f\,:\,\mathbb{N} \to \mathbb{R}$ is measurable with respect to $c$ and that $f$ is integrable over $\mathbb{N}$ with respect to $c$ if and only if the series $\sum_{k=1}^\infty f(k)$ is absolutely convergent in which case
			$$\int_{\mathbb{N}} f\,dc = \sum_{k=1}^\infty\,f(k).$$
	\rule[1.0ex]{17cm}{0.5pt}\
\end{flushleft}

\textbf{\textit{Proof.}}

% https://math.stackexchange.com/questions/1763657/conditions-on-integrable-function-with-counting-measure

% And show the equality: https://math.stackexchange.com/questions/764076/integration-with-respect-to-counting-measure?rq=1

\begin{enumerate}
	\item Use the fact that $f$ is integrable iff $|f|$ is integrable.

	For nonnegative, measurable function $|f|$ in measure space $(\mathbb{N}, \mathcal{M}, c)$, we have
		$$\int_{\mathbb{N}}\,|f|\,dc
		= \sum_{k = 1}^\infty|f(k)|.$$

	Moreover, we also know that $|f|$ is integrable iff $\int_{\mathbb{N}}\,|f|\,dc = \sum_{k = 1}^\infty|f(k)|< +\infty$.

	Thus, every function $f\,:\,\mathbb{N} \to \mathbb{R}$ is measurable with respect to $c$ and that $f$ is integrable over $\mathbb{N}$ with respect to $c$ if and only if the series $\sum_{k=1}^\infty f(k)$ is absolutely convergent.

	\item Following, we will show that
		$$\int_{\mathbb{N}} f\,dc = \sum_{k=1}^\infty\,f(k).$$

	Use the fact that for nonnegative, measurable function $g$ in measure space $(\mathbb{N}, \mathcal{M}, c)$, we have
		$$\int_{\mathbb{N}}\,g\,dc
		= \sum_{k = 1}^\infty g(k).$$

	Since we can write $f$ as $f = f^+ - f^-$ where $f^+,\,f^- \geq 0$, then
		$$\begin{aligned}
		&\int_{\mathbb{N}}\,|f|\,dc
		= \sum_{k = 1}^\infty|f(k)|\\
		\Rightarrow\ &\int_{\mathbb{N}}\,|f^+ - f^-|\,dc
		= \sum_{k = 1}^\infty|f^+(k) - f^-(k)|\\
		\Rightarrow\ &\int_{\mathbb{N}}\,f^+\,dc + \int_{\mathbb{N}}\,f^-\,dc
		= \sum_{k = 1}^\infty f^+(k) + \sum_{k = 1}^\infty f^-(k)\\
		\Rightarrow\ &\int_{\mathbb{N}}\,f^+\,dc - \sum_{k = 1}^\infty f^-(k)
		= \sum_{k = 1}^\infty f^+(k) - \int_{\mathbb{N}}\,f^-\,dc\\
		\Rightarrow\ &\int_{\mathbb{N}}\,f^+\,dc - \int_{\mathbb{N}}\,f^-\,dc
		= \sum_{k = 1}^\infty f^+(k) - \sum_{k = 1}^\infty f^-(k)\\
		\Rightarrow\ &\int_{\mathbb{N}} f\,dc = \sum_{k=1}^\infty\,f(k).
		\end{aligned}$$
\end{enumerate}

%%%%%%%%%%%%%%%%%%%%%%%%%%%%%%%%%%%%%%%%%%%%%%%%%%%%%%%%%%%%%%%%%%%%%%%%%%%

\begin{flushleft}
	\rule[-0.5ex]{17cm}{2pt}\\
		\textbf{Exercise 3}\\
	\rule[1.5ex]{17cm}{0.5pt}
		Let $(X,\mathcal{A},\mu) = (Y,\mathcal{B},\nu) = (\mathbb{N},\mathcal{M},c)$, the measure space defined in the preceding problem. State the Fubini and Tonelli Theorems explicitly for this case.
	\rule[1.0ex]{17cm}{0.5pt}\
\end{flushleft}

\textbf{\textit{Proof.}}

% https://math.stackexchange.com/questions/1799766/relation-between-counting-measure-and-tonelli-theorem

% https://math.stackexchange.com/questions/1709995/verification-prove-the-fubini-tonelli-theorem-when-x-mathcalm-mu-is-an/2936220

Suppose every function $f\,:\,\mathbb{N} \times \mathbb{N} \to \mathbb{R}$ is measurable with respect to $\mu \times \nu$ and $f$ is integrable over $\mathbb{N} \times \mathbb{N}$ with respect to $\mu \times \nu$.

The Tonelli's Theorems for counting measure on $\mathbb{N} \times \mathbb{N}$, if $f$ is $\mathcal{M} \times \mathcal{M}$ and $f \geq 0$, then $\int_\mathbb{N}f\,d\mu$ is measurable with respect to $c$, $\int_\mathbb{N}f\,dc$ is measurable with respect to $\mu$, and
	$$\int_{\mathbb{N}} \left[ \int_{\mathbb{N}}\,f\,d\mu \right]\,d\nu
	= \int_{\mathbb{N} \times \mathbb{N}} f\,d(\mu \times \nu)
	= \int_{\mathbb{N}} \left[ \int_{\mathbb{N}}\,f\,d\nu \right]\,d\mu
	= \sum_{k=1}^\infty \sum_{n=1}^\infty f(k,n).$$

The Fubini's Theorems for counting measure on $\mathbb{N} \times \mathbb{N}$, if $f$ is integrable over $\mathbb{N} \times X$ with repect to $c \times \mu$, that is
	$$\int_{\mathbb{N}} \left[ \int_{\mathbb{N}}\,f\,d\nu \right]\,d\mu
	= \sum_{k=1}^\infty \sum_{n=1}^\infty f(k,n)
	< \infty,$$

then
	$$\int_{\mathbb{N}} \left[ \int_{\mathbb{N}}\,f\,d\mu \right]\,d\nu
	= \int_{\mathbb{N} \times \mathbb{N}} f\,d(\mu \times \nu)
	= \int_{\mathbb{N}} \left[ \int_{\mathbb{N}}\,f\,d\nu \right]\,d\mu
	= \sum_{k=1}^\infty \sum_{n=1}^\infty f(k,n).$$

%%%%%%%%%%%%%%%%%%%%%%%%%%%%%%%%%%%%%%%%%%%%%%%%%%%%%%%%%%%%%%%%%%%%%%%%%%%

\begin{flushleft}
	\rule[-0.5ex]{17cm}{2pt}\\
		\textbf{Exercise 4(i)}\\
	\rule[1.5ex]{17cm}{0.5pt}
		Let $(\mathbb{N},\mathcal{M},c)$ be the measure space defined in \textbf{Exercise 2} and $(X, \mathcal{A}, \mu)$ a general measure space. Consider $\mathbb{N} \times X$ with the product measure $c \times \mu$.

		Show that a subset $E$ of $\mathbb{N} \times X$ is measurable with respect to $c \times \mu$ if and only if for each natural number $k$, $E_k = \{x \in X | (k,x) \in E\}$ is measurable with respect to $\mu$.
	\rule[1.0ex]{17cm}{0.5pt}\
\end{flushleft}

\textbf{\textit{Proof.}}

$(\Rightarrow)$:

We can get from \textbf{Lemma 4.8.2} in Real Analysis by Fon-Che Liu.\\

$(\Leftarrow)$:

We can find some $n \in \mathbb{N}$ such that $E = \cup_n (\{n\} \times E_n)$, since $E_n$ is measurable with respect to $\mu$ and $\{n\} \subset \mathbb{N}$ is measurable with respect to $c$, then $E$ is measurable with respect to $c \times \mu$.\\


%%%%%%%%%%%%%%%%%%%%%%%%%%%%%%%%%%%%%%%%%%%%%%%%%%%%%%%%%%%%%%%%%%%%%%%%%%%

\begin{flushleft}
	\rule[-0.5ex]{17cm}{2pt}\\
		\textbf{Exercise 4(ii)}\\
	\rule[1.5ex]{17cm}{0.5pt}
		Let $(\mathbb{N},\mathcal{M},c)$ be the measure space defined in \textbf{Exercise 2} and $(X, \mathcal{A}, \mu)$ a general measure space. Consider $\mathbb{N} \times X$ with the product measure $c \times \mu$.

		Show that a function $f\,:\,\mathbb{N} \times X \to \mathbb{R}$ is measurable with respect to $c \times \mu$ if and only if for each natural number $k$, $f(k,\cdot)\,:\,X \to \mathbb{R}$ is measurable with respect to $\mu$.
	\rule[1.0ex]{17cm}{0.5pt}\
\end{flushleft}

\textbf{\textit{Proof.}}

$(\Rightarrow)$:

We can get from \textbf{Corollary 4.8.2} in Real Analysis by Fon-Che Liu.\\

% By Tonelli's Theorem, if $f\,:\,\mathbb{N} \times X \to \mathbb{R}$ is measurable with respect to $c \times \mu$, then $f(k,\cdot)\,:\,X \to \mathbb{R}$ is measurable with respect to $\mu$.\\

$(\Leftarrow)$:

Let $f_k = f(k,\cdot)\,:\,X \to \mathbb{R}$. By \textbf{Exercise 4(i)}, we know that if $E_k$ is measurable, then $\{k\} \times E_k$ is measurable. Thus,
	$$f^{-1}((a, \infty]) = \bigcup_k \{k\} \times f_k^{-1}((a,\infty])$$

is a countable union of measurable sets, so $f$ is measurable.\\


%%%%%%%%%%%%%%%%%%%%%%%%%%%%%%%%%%%%%%%%%%%%%%%%%%%%%%%%%%%%%%%%%%%%%%%%%%%

\begin{flushleft}
	\rule[-0.5ex]{17cm}{2pt}\\
		\textbf{Exercise 4(iii)}\\
	\rule[1.5ex]{17cm}{0.5pt}
		Let $(\mathbb{N},\mathcal{M},c)$ be the measure space defined in \textbf{Exercise 2} and $(X, \mathcal{A}, \mu)$ a general measure space. Consider $\mathbb{N} \times X$ with the product measure $c \times \mu$.

		Show that a function $f\,:\,\mathbb{N} \times X \to \mathbb{R}$ is integrable over $\mathbb{N} \times X$ with repect to $c \times \mu$ if and only if for each natural number $k$, $f(k,\cdot)\,:\,X \to \mathbb{R}$ is integrable over $X$ with respect to $\mu$ and
			$$\sum_{k=1}^\infty\int_X|f(k,x)|\,d\mu(x) < \infty.$$
	\rule[1.0ex]{17cm}{0.5pt}\
\end{flushleft}

\textbf{\textit{Proof.}}

Use the fact that $f$ is integrable over $\mathbb{N} \times X$ with repect to $c \times \mu$ iff $|f|$ is integrable over $\mathbb{N} \times X$ with repect to $c \times \mu$.

Also, by \textbf{Exercise 4(ii)}, we have $f$ is measurable with respect to $c \times \mu$, then $|f|$ is measurable and nonnegative, hence, we have
	$$\int_{\mathbb{N}}\,|f|\,dc
	= \sum_{k = 1}^\infty|f(k,x)|.$$

Moreover, $|f|$ is integrable over $\mathbb{N} \times X$ with repect to $c \times \mu$ iff
	$$\int_{\mathbb{N} \times X} |f|\,d(c \times \mu)
	= \int_X \left( \int_{\mathbb{N}} |f|\,dc\right)d\mu
	= \int_X \left( \sum_{k=1}^\infty |f(k,x)| \right) d\mu
	= \sum_{k=1}^\infty \int_X |f(k,x)|\,d\mu
	< +\infty.$$

Since $\int_X |f(k,x)|\,d\mu < \sum_{n=1}^\infty \int_X |f(n,x)|\,d\mu < +\infty$, then $f(k,\cdot)\,:\,X \to \mathbb{R}$ is integrable over $X$ with respect to $\mu$.\\



%%%%%%%%%%%%%%%%%%%%%%%%%%%%%%%%%%%%%%%%%%%%%%%%%%%%%%%%%%%%%%%%%%%%%%%%%%%

\begin{flushleft}
	\rule[-0.5ex]{17cm}{2pt}\\
		\textbf{Exercise 4(iv)}\\
	\rule[1.5ex]{17cm}{0.5pt}
		Let $(\mathbb{N},\mathcal{M},c)$ be the measure space defined in \textbf{Exercise 2} and $(X, \mathcal{A}, \mu)$ a general measure space. Consider $\mathbb{N} \times X$ with the product measure $c \times \mu$.

		Show that if the function $f\,:\,\mathbb{N} \times X \to \mathbb{R}$ is integrable over $\mathbb{N} \times X$ with repect to $c \times \mu$ then
			$$\int_{\mathbb{N} \times X} f\,d(c \times \mu)
			= \sum_{k=1}^\infty\int_X f(k,x)\,d\mu(x)
			< \infty.$$
	\rule[1.0ex]{17cm}{0.5pt}\
\end{flushleft}

\textbf{\textit{Proof.}}\\

Since the function $f\,:\,\mathbb{N} \times X \to \mathbb{R}$ is integrable over $\mathbb{N} \times X$ with repect to $c \times \mu$, by \textbf{Exercise 4(iii)}, we then have
	$$\begin{aligned}
	\int_{\mathbb{N} \times X} f\,d(c \times \mu)
	&= \int_X \left( \int_{\mathbb{N}} f\,dc\right)d\mu\\
	&= \int_X \left( \sum_{k=1}^\infty f(k,x) \right) d\mu\\
	&= \sum_{k=1}^\infty\int_X f(k,x)\,d\mu(x)\\
	&\leq \sum_{k=1}^\infty\int_X|f(k,x)|\,d\mu(x)
	< \infty.
	\end{aligned}$$



%%%%%%%%%%%%%%%%%%%%%%%%%%%%%%%%%%%%%%%%%%%%%%%%%%%%%%%%%%%%%%%%%%%%%%%%%%%

\begin{flushleft}
	\rule[-0.5ex]{17cm}{2pt}\\
		\textbf{Exercise 5}\\
	\rule[1.5ex]{17cm}{0.5pt}
		Let $(X, \mathcal{A}, \mu) = (Y, \mathcal{B}, \nu) = (\mathbb{N}, \mathcal{M}, c)$, the measure space defined in \textbf{Exercise 2}. Define $f: \mathbb{N} \times \mathbb{N} \to \mathbb{R}$ by setting
			$$f(x,y) = \left\{\begin{matrix}
			&2 - 2^{-x} &&\text{if }x = y\\
			& -2 + 2^{-x} &&\text{if }x = y + 1\\
			& 0 &&\text{otherwise.}
			\end{matrix}\right.$$

		Show that $f$ is measurable with respect to the product measure $c \times c$. Also show that
			$$\int_{\mathbb{N}} \left[\int_{\mathbb{N}} f(m,n)\,dc(m)\right]dc(n)
			\neq
			\int_{\mathbb{N}} \left[\int_{\mathbb{N}} f(m,n)\,dc(n)\right]dc(m).$$

		Is this a contradiction either of Fubini's Theorem or Tonelli's Theorem?
	\rule[1.0ex]{17cm}{0.5pt}\
\end{flushleft}

\textbf{\textit{Proof.}}

\begin{enumerate}
	\item Show that $f$ is measurable with respect to the product measure $c \times c$.\\

		Since $\mathbb{N}$ is countable and the only set for which $c \times c$ measure is $0$ is the emptyset, the $c \times c$ measurable sets of $\mathbb{N} \times \mathbb{N}$ are all subsets, hence $f$ is measurable.\\

	\item Show that
			$\int_{\mathbb{N}} \left[\int_{\mathbb{N}} f(m,n)\,dc(m)\right]dc(n)
			\neq
			\int_{\mathbb{N}} \left[\int_{\mathbb{N}} f(m,n)\,dc(n)\right]dc(m).$\\

		Since
			$$\sum_{m = 1, m \in \mathbb{N}}^\infty \sum_{n = 1, n \in \mathbb{N}}^\infty f(m,n)
			= f(1,1) = 1.5,$$

		but
			$$\begin{aligned}
			\sum_{n = 1, n \in \mathbb{N}}^\infty \sum_{m = 1, m \in \mathbb{N}}^\infty f(m,n)
			&= \sum_{n = 1, n \in \mathbb{N}}^\infty (2 - 2^{-n}) + (-2 + 2^{-n-1})\\
			&= \sum_{n = 1, n \in \mathbb{N}}^\infty 2^{-n-1} - 2^{-n}\\
			&= - \sum_{n = 1, n \in \mathbb{N}}^\infty 2^{-n-1}\\
			&= -0.5 \neq 1.5.
			\end{aligned}$$

		So 
			$$\int_{\mathbb{N}} \left[\int_{\mathbb{N}} f(m,n)\,dc(m)\right]dc(n)
			\neq
			\int_{\mathbb{N}} \left[\int_{\mathbb{N}} f(m,n)\,dc(n)\right]dc(m).$$\\

	\item Is this a contradiction either of Fubini's Theorem or Tonelli's Theorem?\\

		No, it doesn't contradict Fubini's theorem or Tonelli's Theorem either.

		Since the measure $c$ is $\sigma$-finite, hence it show we can NOT remove the assumption of non-negativeness or integrability.\\
\end{enumerate}



%%%%%%%%%%%%%%%%%%%%%%%%%%%%%%%%%%%%%%%%%%%%%%%%%%%%%%%%%%%%%%%%%%%%%%%%%%%

\begin{flushleft}
	\rule[-0.5ex]{17cm}{2pt}\\
		\textbf{Exercise 10}\\
	\rule[1.5ex]{17cm}{0.5pt}
		Let $h$ and $g$ be integrable functions on $X$ and $Y$, and define $f(x,y) = h(x)g(y)$. Show that $f$ is integrable on $X \times Y$ with respect to the product measure, then
			$$\int_{X \times Y} f\,d(\mu \times \nu)
			= \int_X h\,d\mu\,\int_Y g\,d\nu.$$
	\rule[1.0ex]{17cm}{0.5pt}\
\end{flushleft}

\textbf{\textit{Proof.}}

\begin{enumerate}
	\item \textbf{Measurability:}

		Use the fact that a function is measurable if and only if it is the pointwise limit of a sequence of simple functions.

		So take simple $a_n,\,b_n$ such that $h(x) = \underset{n \to \infty}{\lim} a_n(x)$ for every $x$ and $g(y) = \underset{n \to \infty}{\lim} b_n(y)$ for every $y$.

		Then
			$$f(x,y) = h(x)g(y) = \underset{n \to \infty}{\lim} a_n(x)b_n(y)$$

		for every $(x,y)$.

		It only remains to observe that each $(x,y) \mapsto a_n(x)b_n(y)$ is a simple function to conclude that $f$ is measurable.

		This follows from
			$$1_X(x) 1_Y(y) = 1_{X \times Y}(x,y).$$\\

	\item \textbf{Integrability:}

		Take two nondecreasing sequences of nonnegative simple functions $a_n(x), b_n(y)$ converging pointwise to $|h(x)|$ and $|g(y)|$ respectively. Then $a_n(x)b_n(y)$ is a nondecreasing sequence of nonnegative simple functions converging pointwise to $|f|$.

		By the monotone convergence theorem
			$$\int_{X \times Y} |f|\,d(\mu\times\nu)
			= \underset{n \to \infty}{\lim} \int_{X \times Y} a_n(x)b_n(y)\,d(\mu\times\nu)(x,y).$$

		By definition of the product measure
			$$\int_{X \times Y} 1_{X \times Y}(x,y)\,d(\mu\times\nu)
			= (\mu\times\nu)(X \times Y)
			= \mu(X) \nu(Y)
			= \int_{X} 1_X(x)\,d\mu \int_{Y} 1_Y(y)\,d\nu.$$

		By linearity, this extends to simple functions. Hence, by monotone convergence again,
			$$\int_{X \times Y} a_n(x)b_n(y)\,d(\mu\times\nu)(x,y)
			= \int_{X} a_n(x)\,d\mu \int_{Y} b_n(y)\,d\nu\
			\longrightarrow\
			\int_X |h|\,d\mu \int_Y |g|\,d\nu.$$

		So $f$ is integrable with
			$$\int_{X \times Y} |f|\,d(\mu\times\nu)
			= \int_X |h|\,d\mu \int_Y |g|\,d\nu.$$

		Note that applying the above to $h^{\pm}$ and $g^{\pm}$, we can deduce
			$$\int_{X \times Y} f\,d(\mu\times\nu)
			= \int_X h\,d\mu \int_Y g\,d\nu.$$
\end{enumerate}




%%%%%%%%%%%%%%%%%%%%%%%%%%%%%%%%%%%%%%%%%%%%%%%%%%%%%%%%%%%%%%%%%%%%%%%%%%%
	% \begin{flushleft}
	% 	\rule[-0.5ex]{17cm}{2pt}\\
	% 		\textbf{Theorem}\\
	% 	\rule[1.5ex]{17cm}{0.5pt}
			
	% 	\rule[1.0ex]{17cm}{0.5pt}\
	% \end{flushleft}
	% \textbf{\textit{Proof.}}\\

\end{document}






