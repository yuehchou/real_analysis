\documentclass[a4paper,11pt]{article}
\usepackage[top=2cm,bottom=2cm,outer=2cm,inner=2cm]{geometry}
\usepackage[utf8]{inputenc}
\usepackage[T1]{fontenc}
\usepackage[inline]{enumitem}
\usepackage{mathrsfs} 
\usepackage{amsfonts}
\usepackage{amsmath}


\title{Real Analysis Homework\\ Chapter 1. Measure theory\\ Due Date: 11/4}
\author{National Taiwan University, Department of Mathematics\\
R06221012 \hspace{0.2cm} Yueh-Chou Lee}
\date{November 4, 2019}
\begin{document}
\maketitle

%%%%%%%%%%%%%%%%%%%%%%%%%%%%%%%%%%%%%%%%%%%%%%%%%%%%%%%%%%%%%%%%%%%%%%%%%%%

\begin{flushleft}
	\rule[-0.5ex]{17cm}{2pt}\\
		\textbf{Exercise 1.27}\\
	\rule[1.5ex]{17cm}{0.5pt}
		Suppose $E_1$ and $E_2$ are a pair of compact sets in $\mathbb{R}^d$ with $E_1 \subset E_2$, and let $a = m(E_1)$ and $b = m(E_2)$. Prove that for any $c$ with $a < c < b$, there is a compact set $E$ with $E_1 \subset E \subset E_2$ and $m(E) = c$.\\
		$[$Hint: As an example, if $d = 1$ and $E$ is a measurable subset of $[0,1]$, consider $m(E \cap [0,t])$ as a function of $t$.$]$
	\rule[1.0ex]{17cm}{0.5pt}\
\end{flushleft}

\textbf{\textit{Proof.}}

Since $E_2 \subset E_1$ and $E_1$, $E_2$ are compact sets in $\mathbb{R}^d$, $E_2 \setminus E_1$ is a bounded and measurable. For any $t > 0$, the sets $(E_2 \setminus E_1) \cap \overline{B_t(0)}$ are also bounded and measurable.

Let
	$$S_t = (E_2 \setminus E_1) \cap \overline{B_t(0)}$$
and define
	$$f(t) = m(S_t).$$
Hence, if we can prove $f$ is a continuous function, the proof will be done.\\

Let $0 \leq \tau < t$. Notice that the function $|f(t) - f(\tau)| = |m(S_t) - m(S_\tau)|$.

Since $S_\tau \subset S_t$,
	$$|m(S_t) - m(S_\tau)| = m(S_t) - m(S_\tau) = m(S_t \setminus S_\tau).$$

Now, notice that
	$$(S_t \setminus S_\tau) \subset \overline{B_t(0)} \setminus \overline{B_\tau(0)}$$
so
	$$m(S_t \setminus S_\tau)
	\leq m(\overline{B_t(0)} \setminus \overline{B_\tau(0)})
	\leq \alpha(d)\,(t^d - \tau^d)$$
where $\alpha(d)\,=\,\dfrac{\pi^{\frac{d}{2}}}{\Gamma(\frac{d}{2} + 1)}$, the volume of the $d$-dimensional unit ball.(Exercise 1.6)\\

Notice that the function $g(t) = \alpha\,t^d$ is continuous, but not uniformly continuous. Thus,
	$$|t - \tau| < \delta \,\Rightarrow\,\alpha(d)\,(t^d - \tau^d) \leq \varepsilon(t).$$

Therefore, $f$ is a continuous function, and by the Intermediate Value Theorem, given $c$ with $a < c < b$, we can find a $t^* > 0$ such that $m(E) = m(E_1 \cup S_{t^*}) = c$, where $E_1 \cup S_{T^*}$ is compact.\\\\


%%%%%%%%%%%%%%%%%%%%%%%%%%%%%%%%%%%%%%%%%%%%%%%%%%%%%%%%%%%%%%%%%%%%%%%%%%%

\begin{flushleft}
	\rule[-0.5ex]{17cm}{2pt}\\
		\textbf{Exercise 1.28}\\
	\rule[1.5ex]{17cm}{0.5pt}
		Let $E$ be a subset of $\mathbb{R}$ with $m_*(E) > 0$. Prove that for each $0 < \alpha < 1$, there exists an open interval $I$ so that 
			$$m_*(E \cap I) \geq \alpha\,m_*(I).$$
		Loosely speaking, this estimate shows that $E$ contains almost a whole interval.\\
		$[$Hint: Choose an open set $\mathcal{O}$ that contains $E$, and such that $m_*(E) \geq \alpha\,m_*(\mathcal{O})$. Write $\mathcal{O}$ as the countable union of disjoint open intervals, and show that one of these intervals must satisfy the desired property.$]$
	\rule[1.0ex]{17cm}{0.5pt}\
\end{flushleft}

\textbf{\textit{Proof.}}

Let $E$ be a subset of $\mathbb{R}$ with $m_*(E) > 0$ and fix an $\alpha \in (0,1)$. Because $E$ has positive outer measure, we can find a covering of $E$ by closed and almost disjoint interval $I_j$ such that
	$$\sum_j |I_j| < m_*(E) + \frac{\varepsilon}{2}.$$

We can expand each of these $I_j$ to an open cube $I'_j$ such that
	$$m_*(I'_j - Q_j) < \frac{\varepsilon}{2^{k+1}}$$
and set $\mathcal{O} = \cup_j Q'_j$. So $\mathcal{O}$ is an open set containing $E$ and so we can write
	$$E = E \cap \mathcal{O} = \cup_j\,E \cap I'_j$$

By monotonicity we can see that $m_*(E) \leq \sum_j m_*(E \cap I'_j)$.\\

Now, suppose towards a contradiction that for every $j \in \mathbb{Z}^+$, we have that $m_*(E \cap I'_j) < \alpha\,m_*(I'_j)$. Then
	$$m_*(E)
	\leq \sum_j m_*(E \cap I'_j)
	< \alpha \sum_j m_*(I'_j)
	< \alpha(m_*(E) + \varepsilon)$$

But, if we take
	$$\varepsilon < \frac{1 - \alpha}{\alpha}\,m_*(E)$$

Then we would get that $m_*(E) < m_*(E)$, which is impossible. Hence, we must be able to find some $j$ such that
	$$m_*(E \cap I'_j) \geq \alpha\,m_*(I)$$\\\\

%%%%%%%%%%%%%%%%%%%%%%%%%%%%%%%%%%%%%%%%%%%%%%%%%%%%%%%%%%%%%%%%%%%%%%%%%%%

\begin{flushleft}
	\rule[-0.5ex]{17cm}{2pt}\\
		\textbf{Exercise 1.29}\\
	\rule[1.5ex]{17cm}{0.5pt}
		Suppose $E$ is a measurable subset of $\mathbb{R}$ with $m(E) > 0$. Prove that the \textbf{difference set} of $E$, which is defined by
			$$\{z \in \mathbb{R}\,:\,z = x - y \text{ for some }x,\,y\in E\},$$
		contains an open interval centered at the origin.\\

		If $E$ contains an interval, then the conclusion is straightforward. In general, one may rely on Exercise 1.28.\\

		% $[$Hint: Indeed, by Exercise 1.28, there exists an open interval $I$ so that $m(E \cap I) \geq \frac{9}{10}\,m(I)$. If we denote $E \cap I$ by $E_0$, and suppose that the difference set of $E_0$ does not contain an open interval around the origin, then for arbitrarily small $a$ the sets $E_0$, and $E_0 + a$ are disjoint. From the fact that $(E_0 \cup (E_0 + a)) \subset (I \cup (I + a))$ we get a contradiction, since the left-hand side has measure $2m(E_0)$, while the right-hand side has measure only slightly larger than $m(I)$.$]$\\

		A more general formulation of this result is as follows.
	\rule[1.0ex]{17cm}{0.5pt}\
\end{flushleft}

\textbf{\textit{Proof.}}

It is enough to prove the claim for a measurable subset of $E$ with positive measure, so we do some reductions by finding some nice subsets of $E$ like this and replacing $E$ with these subsets.\\

First, since the collection $\{E \cap (n,\,n+1]\,:\,n\in \mathbb{Z}\}$ of disjoint measurable sets cover $E$.

By countable additivity, there exists $n \in \mathbb{N}$ such that $m(E \cap (n,\,n+1])) > 0$.

So we may assume that $E$ has finite measure.\\

Second, since $0 < m(E) < \infty$, by \textbf{Theorem 3.4 (iii)} of the textbook, there exists a compact set $K$ contained in $E$ such that choose $\varepsilon = \frac{m(E)}{2}$ then
	$$m(E \setminus K) \leq \frac{m(E)}{2}.$$

Then by addivity of the measure we get $m(K) \geq \frac{m(E)}{2} > 0$.

So we may assume that $E$ is compact.\\

Third, since $0 < m(E) < \infty$, by \textbf{Theorem 3.4 (iii)} of the textbook again, there exists an open set $U$ containing $E$ such that
	$$m(U \setminus E) \leq \frac{m(E)}{2}.$$
Hence, $m(U) \leq \frac{3}{2}\,m(E) < 2m(E)$.\\

Now $E$ and $U^c$ are disjoint sets where $E$ is compact and $U^c$ is closed. Therefore $\delta := d(E,\,U^c) > 0$.

We claim that $(-\delta,\,\delta)$ lies in the difference set of $E$.

So let $t \in (-\delta,\,\delta)$. Then by the definition of $\delta$, the set
	$$E + t = \{x + t\,:\,x \in E\}$$
does not intersect $U^c$, therefore $E + t \subseteq U$ and hence $(E + t) \cup E \subseteq U$.\\

Note that $E + t$ is a measurable set with $m(E + t) = m(E)$.

Suppose $(E + t) \cap E = \emptyset$. Then by additivity, we get
	$$m(U) \geq m((E + t) \cup E) = 2m(E)$$
which is contradiction.\\

Thus, there exists $x,\,y \in E$ such that$ x + t = y$, so $t \in (-\delta,\,\delta)$ lies in the difference set of $E$.\\\\\\



%%%%%%%%%%%%%%%%%%%%%%%%%%%%%%%%%%%%%%%%%%%%%%%%%%%%%%%%%%%%%%%%%%%%%%%%%%%

\begin{flushleft}
	\rule[-0.5ex]{17cm}{2pt}\\
		\textbf{Exercise 1.31}\\
	\rule[1.5ex]{17cm}{0.5pt}
		The result in Exercise 1.29 provides an alternate proof of the non-measurability of the set $\mathcal{N}$ studied in the text. In fact, we may also prove the non-measurability of a set in $\mathbb{R}$ that is very closely related to the set $\mathcal{N}$.\\

		Given two real numbers $x$ and $y$, we shall write as before that $x \sim y$ whenever the difference $x-y$ is rational. Let $\mathcal{N}^*$ denote a set that consists of one element in each equivalence class of $\sim$. Prove that $\mathcal{N}^*$ is non-measurable by using the result in Exercise 1.29.\\

		$[$Hint: If $\mathcal{N}^*$ is measurable, then so are its translates $\mathcal{N}^*_n = \mathcal{N}^* + r_n$, where $\{r_n\}_{n = 1}^\infty$ is an enumeration of $\mathbb{Q}$. How does this imply that $m(\mathcal{N}^*) > 0$? Can the difference set of $\mathcal{N}^*$ contain an open interval centered at the origin?$]$
	\rule[1.0ex]{17cm}{0.5pt}\
\end{flushleft}

\textbf{\textit{Proof.}}

We proceed by contradiction, assuming that $\mathcal{N}^*$ is measurable.

If we consider as indicated in the hint the set $\mathcal{N}_N^*$, which is the transaltion of the set $\mathcal{N}^*$ by $r_n$, then since by construction our equivalence classes are real numbers that differ by a rational, we can enumerating all of $\mathbb{Q}$ take the countable union of these transalte, which have same measure.\\

Noe if we take
	$$m\left( \cup_{n=1}^{\infty}\, \mathcal{N}^* + r_n \right)
	= m(\mathbb{R})
	= \infty,$$

and by countable sub-additivity, we know that
	$$m\left( \cup_{n=1}^{\infty}\, \mathcal{N}^* + r_n \right)
	\leq \sum_{n=1}^\infty\,m(\mathcal{N}^* + r_n)
	= \sum_{i=1}^\infty m(C_n^*)
	= \infty$$

from the previous result.\\

On the other hand, we have if $\mathcal{N}^*$ is measurable that for sets of positive measure, then the difference set of $\mathcal{N}^*$ contains an open interval centered at the origin. But this will not happen, since then $x - y = 0$ would imply $x \sim y$, and since the set consists of the selection of one element (using axiom of choice) from each equivalent class, then $0$ is not in the difference set; we infer that $m(\mathcal{N}^*) = 0$, so together this show that actually, $\mathcal{N}^*$ is not measurable.\\\\\\


%%%%%%%%%%%%%%%%%%%%%%%%%%%%%%%%%%%%%%%%%%%%%%%%%%%%%%%%%%%%%%%%%%%%%%%%%%%

\begin{flushleft}
	\rule[-0.5ex]{17cm}{2pt}\\
		\textbf{Exercise 1.32}\\
	\rule[1.5ex]{17cm}{0.5pt}
		Let $\mathcal{N}$ denote the non-measurable subset of $I = [0, 1]$ constructed at the end of Section 1.3.
		\begin{enumerate}
			\item [(a)] Prove that if $E$ is a measurable subset of $\mathcal{N}$ , then $m(E) = 0$.

			\item [(b)] If $G$ is a subset of $\mathbb{R}$ with $m_*(G) > 0$, prove that a subset of $G$ is nonmeasurable.
		\end{enumerate}
		$[$Hint: For (a) use the translates of $E$ by the rationals.$]$
	\rule[1.0ex]{17cm}{0.5pt}\
\end{flushleft}

\textbf{\textit{Proof.}}

\begin{enumerate}
	\item [(a)]
		Let $\{r_k\}_{k=1}^\infty$ be an enumeration of the rationals in the interval $[-1,\,1]$ and let $E_k = E + r_k$ for each $k$.\\

		Since $E \subseteq \mathcal{N}$, $E_k \subseteq N_k$. Since each of the $\mathcal{N}_k$ are pairwise disjoint, each of the $E_k$ are pairwise disjoint.\\

		Now, the Lebesgue measure is translation invariant, so $m(E_k) = m(E)$ for each $k$. We also have that $\cup_{k = 1}^\infty\,E_k \subseteq \cup_{k=1}^\infty\,\mathcal{N}_k \subseteq [-1,2]$. It follows that
			$$\sum_{k=1}^\infty\,m(E_k) = m(\cup_{k=1}^\infty\,E_k) \leq 3 \quad \quad (\text{since } \cup_{k=1}^\infty\,E_k \subseteq [-1,\,2])$$

		But $m(E_k) = m(E)$ for each $k$. Hence
			$$3 \geq \sum_{k=1}^\infty\,m(E_k) = \sum_{k=1}^\infty\,m(E)$$
		which implies that $m(E) = 0$.\\

	\item [(b)]
		Since $m(G) > 0$, we can find for any $\varepsilon > 0$ a closed interval $[a,\,b] \subseteq G$ with $m(G \setminus [a,\,b]) \leq \varepsilon$.\\

		Now, consider the set $G - a$ ($G$ translated by $-a$ units). Since the Lebesgue measure is translation invariant, $m(G - a) = m(G)$.\\

		Furthermore, the interval $[0,\,b-a] \subseteq G-a$. Let $A = [0,\,b-a] \cap \mathcal{N}$.

		Observe that $A \subseteq G$. Suppose $A$ is measurable $A$ is measurable. Since $A \subseteq \mathcal{N},\,m(A) = 0$ by part $(a)$. It follows that
			$$m(G) = m(G - a) = m((G - a) \setminus A) + m(A) \leq \varepsilon + 0 = \varepsilon$$

		Since $\varepsilon$ can be chosen arbitrarily small, we conclude that $m(G) = 0$, which is a contradiction. Hence, it must be that $A$ is non-measurable.\\\\\\

\end{enumerate}


%%%%%%%%%%%%%%%%%%%%%%%%%%%%%%%%%%%%%%%%%%%%%%%%%%%%%%%%%%%%%%%%%%%%%%%%%%%

\begin{flushleft}
	\rule[-0.5ex]{17cm}{2pt}\\
		\textbf{Exercise 1.33}\\
	\rule[1.5ex]{17cm}{0.5pt}
		Let $\mathcal{N}$ denote the non-measurable set constructed in the text. Recall from the exercise above that measurable subsets of $\mathcal{N}$ have measure zero.\\

		Show that the set $\mathcal{N}^c = I - \mathcal{N}$ satisfies $m_*(\mathcal{N}^c) = 1$, and conclude that if $E_1 = \mathcal{N}$ and $E_2 = \mathcal{N}^c$, then
			$$m_*(E_1) + m_*(E_2) \neq m_*(E_1 \cup E_2),$$
		although $E_1$ and $E_2$ are disjoint.\\

		$[$Hint: To prove that $m_*(\mathcal{N}^c) = 1$, argue by contradiction and pick a measurable set $U$ such that $U \subset I,\,\mathcal{N}^c \subset U$ and $m_*(U) < 1 - \varepsilon.]$
	\rule[1.0ex]{17cm}{0.5pt}\
\end{flushleft}

\textbf{\textit{Proof.}}

\begin{enumerate}

	\item [(i)]

		Suppose $m_*(\mathcal{N}^c) < 1$, so there exists $\varepsilon > 0$ such that $m_*(\mathcal{N}^c) < 1 - \varepsilon$. Since
			$$m_*(\mathcal{N}^c) = \inf\,\{m(U)\,:\,\mathcal{N}^c \subseteq U-\text{open}\}$$
		there exists an open, hence measurable set $U$ containing $\mathcal{N}^c$ such that $m(U) < 1 - \varepsilon$.\\

		Note that $U \cap I$ is also a measurable containing $\mathcal{N}^c$ with $m(U \cap I) < 1 - \varepsilon$, so we may assume $U \subseteq I$.\\

		Since $\mathcal{N}^c = I - \mathcal{N} \subseteq U \subseteq I$, we have $I - U \subset \mathcal{N}$. But $I - U$ is a measurable set so by additivity of the measure, we have
			$$m(I - U) = m(I) - m(U) = 1 - m(U) > \varepsilon.$$

		So $I - U$ is a measurable subset of $\mathcal{N}$ with positive measure; a contradiction.\\

		Therefore, $m_*(\mathcal{N}^c) \geq 1$, but on the other hand, $m_*(\mathcal{N}^c) \leq m_*(I) = 1$, hence, $m_*(\mathcal{N}^c) = 1$.\\

	\item [(ii)]
		We know that sets with outer measure zero are measurable, so $m_*(\mathcal{N}) > 0$.

		Thus, by part (i), we have
			$$m_*(\mathcal{N}) + m_*(\mathcal{N}^c) > 1 = m_*(I) = m_*(\mathcal{N} \cup \mathcal{N}^c).$$\\\\

\end{enumerate}


%%%%%%%%%%%%%%%%%%%%%%%%%%%%%%%%%%%%%%%%%%%%%%%%%%%%%%%%%%%%%%%%%%%%%%%%%%%

\begin{flushleft}
	\rule[-0.5ex]{17cm}{2pt}\\
		\textbf{Exercise 1.34}\\
	\rule[1.5ex]{17cm}{0.5pt}
		Let $\mathcal{C}_1$ and $\mathcal{C}_2$ be any two Cantor sets (constructed in Exercise 1.3). Show that there exists a function $F\,:\,[0, 1] \to [0, 1]$ with the following properties:
		\begin{enumerate}
			\item [(i)] $F$ is continuous and bijective,

			\item [(ii)] $F$ is monotonically increasing,

			\item [(iii)] $F$ maps $\mathcal{C}_1$ surjectively onto $\mathcal{C}_2$.
		\end{enumerate}
		$[$Hint : Copy the construction of the standard Cantor-Lebesgue function.$]$
	\rule[1.0ex]{17cm}{0.5pt}\
\end{flushleft}

\textbf{\textit{Proof.}}

Let $\mathcal{C}$ be a Cantor set of constant dissection as in \textbf{Exercise 1.3}. By construction, $\mathcal{C}$ is the intersection of a family $\{C_n\}_{n \in \mathbb{N}}$ of closed sets where each $\mathcal{C}_n$ is disjoint union of $2^n$ closed intervals. So we can label these $2^n$ intervals from left to right by bit strings of length $n$, that is, words of length $n$ consisting of $0$'s and $1$'s.\\

For example, $\mathcal{C}_1 = I_0 \cup I_1$ where $I_0$ is the interval on the left hand side in $\mathcal{C}_1$ and $I_1$ is the one on the right. Keeping the labeling in a lexicographic order, we have $\mathcal{C}_2 = I_{00} \cup I_{01} \cup I_{10} \cup I_{11}$ and in general $\mathcal{C}_n$ is the union of $I_b$'s where \textbf{b}'s vary over length $n$ bit strings.\\

Note that $I_b \subseteq I_c$ if and only if \textbf{c} can be truncated from the right to obtain \textbf{b}. For example, $I_0 \supseteq I_{01} \supseteq I_{010} \supseteq I_{0100}$. In general given an infinite sequence $\textbf{a} = (a_n)$ of 0's and 1's, if we write $\textbf{a}|_n$ for its $n$-truncation $(a_1,\,\dots,\,a_n)$, there is a decreasing sequence
	$$I_{\textbf{a}|_1} \supseteq I_{\textbf{a}|_2} \supseteq I_{\textbf{a}|_3} \cdots$$

By compactness, the intersection
	$$\cap_{n \in \mathbb{N}} I_{\textbf{a}|_n}$$
which lies in $\mathcal{C}$, is nonempty.\\

Yet the diameter of the intersection is zero, hence it must be a singleton. Therefore every infinite sequence \textbf{a} of 0’s and 1’s uniquely determines a point in $\mathcal{C}$. So we get a map
	$$f\,:\,\{0,\,1\}^{\mathbb{N}} \to \mathcal{C}$$
which is surjective since points in $\mathcal{C}$ by definition survives the intersection of $\mathcal{C}_n$’s, hence, lie in infinitely many (hence in an infinite decreasing chain of) $I_{\textbf{b}}$’s.\\

If two infinite sequences are distinct, they have different truncations so as the intervals get finer, the two points these sequences determine will fall into different intervals. Hence $f$ is a bijection.\\

Two points in $\mathcal{C}$ lie in the same $I_{\textbf{b}}$ where \textbf{b} is a finite bit string if and only if their inverse images under $f$ both start with \textbf{b}. It follows from this observation (as we did for the middle thirds Cantor set in \textbf{Exercise 1.2}) that $f$ is continuous. And since $f$ goes from a compact space to a Hausdorff space, $f$ is a homeomorphism.\\

Also, observe that if we order $\{0,\,1\}^{\mathbb{N}}$ by lexicographic ordering, then $f$ preserves the order. Because if a sequence beats another sequence lexicographically, then at some point it will lie to the right side of a dissection while the other lies on the left side.\\

So if $\mathcal{C}_1$ and $\mathcal{C}_2$ are Cantor sets, we have order preserving homeomorphisms $f_1\,:\,\{0,\,1\}^{\mathbb{N}} \to \mathcal{C}_1$ and $f_2\,:\,\{0,\,1\}^{\mathbb{N}} \to \mathcal{C}_2$; thus, $f_2 \circ f_1^{-1}$ gives an order preserving homeomorphism from $\mathcal{C}_1$ to $\mathcal{C}_2$.\\\\\\



%%%%%%%%%%%%%%%%%%%%%%%%%%%%%%%%%%%%%%%%%%%%%%%%%%%%%%%%%%%%%%%%%%%%%%%%%%%

\begin{flushleft}
	\rule[-0.5ex]{17cm}{2pt}\\
		\textbf{Exercise 1.35}\\
	\rule[1.5ex]{17cm}{0.5pt}
		Give an example of a measurable function $f$ and a continuous function $\Phi$ so that $f \circ \Phi$ is non-measurable.\\

		$[$Hint: Let $\Phi\,:\,\mathcal{C}_1 \to \mathcal{C}_2$ as Exercise 1.34, with $m(\mathcal{C}_1) > 0$ and $m(\mathcal{C}_2) = 0$. Let $N \subset \mathcal{C}_1$ be non-measurable, and take $f = \chi_{\Phi(N)}.]$\\

		Use the construction in the hint to show that there exists a Lebesgue measurable set that is not a Borel set.
	\rule[1.0ex]{17cm}{0.5pt}\
\end{flushleft}

\textbf{\textit{Proof.}}

Follow the hint, let $\Phi\,:\,\mathcal{C}_1 \to \mathcal{C}_2$ as Exercise 1.34, with $m(\mathcal{C}_1) > 0$ and $m(\mathcal{C}_2) = 0$.

Let $N \subset \mathcal{C}_1$ be non-measurable, and take $f = \chi_{\Phi(N)}$. We know such $N$ exists by Exercise 1.32(b).\\

Since $\Phi(N) \subseteq \mathcal{C}_2$ and $m(\mathcal{C}_2) = 0$, we have $m_*(\Phi(N)) = 0$ and so $\Phi(N)$ is a measurable set.\\

Therefore, $f$ is a measurable function. However,
	$$(f \circ \Phi)^{-1} (1)
	= \Phi^{-1}(f^{-1}(\{ 1 \}))
	= \Phi^{-1}(\Phi(N))
	= N$$
is not measurable, hence $f \circ \Phi$ is not a measurable function.\\

Also, the measurable set $\Phi(N)$ cannot be Borel because the inverse images of Borel sets under continuous functions are Borel, but although $\Phi$ is continuous, $\Phi^{-1}(\Phi(N)) = N$ is not even measurable.\\\\\\


%%%%%%%%%%%%%%%%%%%%%%%%%%%%%%%%%%%%%%%%%%%%%%%%%%%%%%%%%%%%%%%%%%%%%%%%%%%

\begin{flushleft}
	\rule[-0.5ex]{17cm}{2pt}\\
		\textbf{Exercise 1.37}\\
	\rule[1.5ex]{17cm}{0.5pt}
		Suppose $\Gamma$ is a curve $y = f(x)$ in $\mathbb{R}^2$, where $f$ is continuous. Show that $m(\Gamma) = 0$.\\

		$[$Hint: Cover $\Gamma$ by rectangles, using the uniform continuity of $f.]$
	\rule[1.0ex]{17cm}{0.5pt}\
\end{flushleft}

\textbf{\textit{Proof.}}

Note that since the map $x \mapsto -x$ preserves areas of rectangles, $\Gamma$ has the same measure with the curve given by $y = |f(x)|$. Therefore we may assume $f$ is nonnegative. Also since
	$$\Gamma = \cup_{n \in \mathbb{N}}\,\{(x,\,f(x))\,:\,x\in[n,\,n+1]\}$$
and measure is countably sub-additive, it suffices to show that each term in the above union has measure zero.\\

Thus, we may assume that $f\,:\,[a,\,b] \to \mathbb{R}$ where $[a,\,b] \subseteq \mathbb{R}$ is a finite interval. Moverover, by replacing $f$ with $f + 1$, we may assume that $f(x) \geq 1$ for every $x \in [a,\,b]$. Then given $0 < \varepsilon < 1$, the set
	$$E_\varepsilon = \{(x,\,y)\,:\,a \leq x \leq b,\,f(x) - \varepsilon \leq y \leq f(x) + \varepsilon\}$$
contains $\Gamma$.\\

But since $f \geq 1 > \varepsilon$, both $f + \varepsilon$ and $f - \varepsilon$ are nonnegative and continuous, therefore, the measure of $E_\varepsilon$ can be calculated by a definite Riemann integral as
	$$m(E_\varepsilon)
	= \int_a^b\,(f(x) + \varepsilon)\,dx - \int_a^b\,(f(x) - \varepsilon)\,dx
	= \int_a^b\,2\varepsilon\,dx
	= 2\varepsilon(b - a).$$

So $m(\Gamma) \leq 2\varepsilon(b - a)$ for arbitrarily small $\varepsilon$. As $a,\,b$ is independent from $\varepsilon$, this shows that $m(\Gamma) = 0$.\\\\\\


%%%%%%%%%%%%%%%%%%%%%%%%%%%%%%%%%%%%%%%%%%%%%%%%%%%%%%%%%%%%%%%%%%%%%%%%%%%

\begin{flushleft}
	\rule[-0.5ex]{17cm}{2pt}\\
		\textbf{Exercise 1.38}\\
	\rule[1.5ex]{17cm}{0.5pt}
		Prove that $(a + b)^\gamma \geq a^\gamma + b^\gamma$ whenever $\gamma \geq 1$ and $a, b \geq 0$. Also, show that the reverse inequality holds when $0 \leq \gamma \leq 1$.\\

		$[$Hint: Integrate the inequality between $(a + t)^{\gamma - 1}$ and $t^{\gamma - 1}$ from $0$ to $b.]$
	\rule[1.0ex]{17cm}{0.5pt}\
\end{flushleft}

\textbf{\textit{Proof.}}

\begin{enumerate}
	\item [(i)]
		For all $a,\,b,\,t \geq 0$ and $\gamma \geq 1$, we have
			$$\begin{aligned}
			(a + t)^{\gamma - 1} \geq t^{\gamma - 1}\,
			&\Rightarrow\, \int_0^b\,(a + t)^{\gamma - 1}\,dt\,
			\geq\,\int_0^b\,t^{\gamma - 1}\,dt\\
			&\Rightarrow\,\frac{1}{\gamma}\left[(a + b)^\gamma - a^\gamma \right]\,\geq\,\frac{1}{\gamma}\,b^\gamma\\
			&\Rightarrow\,(a + b)^\gamma\,\geq\,a^\gamma + b^\gamma
			\end{aligned}$$

	\item [(ii)]
		For all $a,\,b \geq 0$ and $\gamma \in [0,\,1]$. Let $k = \frac{1}{\gamma}$, $c = a^{\frac{1}{k}} = a^\gamma$ and $d = b^{\frac{1}{k}} = b^\gamma$. Then
			$$\begin{aligned}
			(a + b)^\gamma \leq a^\gamma + b^\gamma\,
			&\Leftrightarrow\,
			(c^k + d^k)^{\frac{1}{k}} \leq c + d\\
			&\Leftrightarrow\,
			c^k + d^k \leq (c + d)^k\\
			&\Leftrightarrow\,
			1 + \left(\frac{d}{c}\right)^k \leq \left( 1 + \frac{d}{c} \right)^k\\
			&\Leftrightarrow\,
			\left(\frac{d}{c}\right)^k \leq \left(1 + \frac{d}{c} \right)^k - 1\\
			&\Leftrightarrow\,
			k \int_{0}^{\frac{d}{c}}\,t^{k - 1}\,dt \leq k \int_0^{\frac{d}{c}}\,(1 + t)^{k - 1}\,dt
			\end{aligned}$$
\end{enumerate}


%%%%%%%%%%%%%%%%%%%%%%%%%%%%%%%%%%%%%%%%%%%%%%%%%%%%%%%%%%%%%%%%%%%%%%%%%%%
	% \begin{flushleft}
	% 	\rule[-0.5ex]{17cm}{2pt}\\
	% 		\textbf{Theorem}\\
	% 	\rule[1.5ex]{17cm}{0.5pt}
			
	% 	\rule[1.0ex]{17cm}{0.5pt}\
	% \end{flushleft}
	% \textbf{\textit{Proof.}}\\

\end{document}






