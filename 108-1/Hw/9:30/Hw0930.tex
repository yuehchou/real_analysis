\documentclass[a4paper,11pt]{article}
\usepackage[top=2cm,bottom=2cm,outer=2cm,inner=2cm]{geometry}
\usepackage[utf8]{inputenc}
\usepackage[T1]{fontenc}
\usepackage[inline]{enumitem}
\usepackage{mathrsfs} 
\usepackage{amsfonts}
\usepackage{amsmath}


\title{Real Analysis Homework\\ Chapter 1. Measure theory\\ Due Date: 10/2}
\author{National Taiwan University, Department of Mathematics\\
R06221012 \hspace{0.2cm} Yueh-Chou Lee}
\date{October 2, 2019}
\begin{document}
\maketitle

\begin{flushleft}
	\rule[-0.5ex]{17cm}{2pt}\\
		\textbf{Exercise 1.1:}\\
	\rule[1.5ex]{17cm}{0.5pt}
		Prove that the Cantor set $\mathcal{C}$ constructed in the text is totally disconnected and perfect. In other words, given two distinct points $x, y \in \mathcal{C}$, there is a point $z \notin \mathcal{C}$ that lies between $x$ and $y$, and yet $\mathcal{C}$ has no isolated points.
	\rule[1.0ex]{17cm}{0.5pt}\
\end{flushleft}

\textbf{\textit{Proof.}}

\begin{enumerate}
	\item \textbf{Disconnected:}

	Let $x, y \in \mathcal{C}$ and $x \neq y$. Then $x, y \in \mathcal{C_k}$ for all $k \in \mathbb{N}$.

	Since $x \neq y$, we can find $N \in \mathbb{N}$ such that $\dfrac{1}{3^N} < |x - y|$. Hence, $x$ and $y$ belong to different intervals of $\mathcal{C_N}$.

	By the construction of the Cantor set, there must be at least one interval between $x$ and $y$ which does not belong to $\mathcal{C_N}$ , and so does not belong to $\mathcal{C}$.

	Select one such interval. Choosing any point $z$ in this interval satisfies that $z$ lies between $x$ and $y$ and $z \notin \mathcal{C}$. Therefore, $\mathcal{C}$ is totally disconnected.\\


	\item \textbf{Perfect:}

	let $\varepsilon > 0$ be given and consider $B(x, \varepsilon)$ for any $x \in \mathcal{C}$. Let $I_k$ denote the interval to which $x$ belongs in $C_k$. We can find $N \in \mathbb{N}$ such that $I_N \subset B(x, \varepsilon)$.

	Now, this interval must have two endpoints $a_N$ and $b_N$. By the construction of the Cantor set, we know that the endpoints of any interval are never removed, and so $a_N, b_N \in \mathcal{C}$. Furthermore, we have that $a_N, b_N \in I_N \subset B(x, \varepsilon)$. Therefore, $x$ is not isolated.\\\\
\end{enumerate}


\begin{flushleft}
	\rule[-0.5ex]{17cm}{2pt}\\
		\textbf{Exercise 1.2(a):}\\
	\rule[1.5ex]{17cm}{0.5pt}
		The Cantor set $\mathcal{C}$ can also be described in terms of ternary expansions.\\

		Every number in $[0,1]$ has a ternary expansion
			$$x = \overset{\infty}{\underset{k = 1}{\sum}} a_k 3^{-k},
			\quad \quad \quad \text{where } a_k = 0, 1, \text{or }2.$$
		Prove that $x \in \mathcal{C}$ if and only if $x$ has a representation as above where every $a_k$ is either $0$ or $2$.
	\rule[1.0ex]{17cm}{0.5pt}\
\end{flushleft}

\textbf{\textit{Proof.}}

$(\Rightarrow)$

Let $x \in \mathcal{C}$. Consider $\mathcal{C}_1$. It must
be that x belongs to one of $[0,\dfrac{1}{3}]$ or $[\dfrac{2}{3}, 1]$.

Next, consider $\mathcal{C}_2$. The interval of $\mathcal{C}_1$ to which $x$ currently belongs will be divided into three subintervals, and so we append $0$ to the ternary expansion of $x$ if it belongs to the leftmost subinterval or $2$ if it belongs to the rightmost subinterval. 

Continuing in this way, we see that $x$ has an associated ternary expansion containing only the digits $0$ and $2$.\\

$(\Leftarrow)$

Let
	$$x = \overset{\infty}{\underset{k = 1}{\sum}} a_k 3^{-k},
	\quad \quad \quad \text{where } a_k = 0\text{ or }2.$$

If $a_1 = 0$, we choose the left subinterval of $\mathcal{C}_1$. If $a_1 = 2$, we choose the rightmost subinterval of $\mathcal{C}_1$.

When we form $\mathcal{C}_2$, the interval we have just chosen will be subdivided into three subintervals. If $a_2 = 0$, we select the leftmost subinterval. If $a_2 = 2$, we select the rightmost subinterval.

Continue in this way. Since the length of these intervals can be made arbitrarily small, we see that the ternary expansion of $x$ uniquely specifies its location on the real line.\\\\

\begin{flushleft}
	\rule[-0.5ex]{17cm}{2pt}\\
		\textbf{Exercise 1.2(b):}\\
	\rule[1.5ex]{17cm}{0.5pt}
		The \textbf{Cantor-Lebesgue function} is defined on $\mathcal{C}$ by
			$$F(x) = \overset{\infty}{\underset{k = 1}{\sum}} \frac{b_k}{2^k}
			\quad \quad \text{ if }
			x = \overset{\infty}{\underset{k = 1}{\sum}} a_k 3^{-k},
			\quad \text{ where } b_k = \frac{a_k}{2}.$$
		In this definition, we choose the expansion of $x$ in which $a_k = 0$ or $2$.

		Show that $F$ is well defined and continuous on $\mathcal{C}$, and moreover $F(0) = 0$ as well as $F(1) = 1$.
	\rule[1.0ex]{17cm}{0.5pt}\
\end{flushleft}

\textbf{\textit{Proof.}}

Let $x, x' \in \mathcal{C}$ with $x = x'$. Denote the $k$-th digit of the ternary expansion of $x$ and $x'$ by $a_k$ and $a_k'$, respectively.\\

Suppose $a_k \neq a_k'$ for all $k$. Then $a_N \neq a_N'$ for some $N$. From the construction in the part (a), we see that $x$ and $x'$ must belong to different subintervals in $\mathcal{C}_N$, and so $x \notin x'$, which is a construction.\\

Now, let $b_k = \dfrac{a_k}{2}$ and $b_k' = \dfrac{a_k'}{2}$. Then $b_k = b_k'$ for all $k$. Hence
	$$F(x)
	= \overset{\infty}{\underset{k = 1}{\sum}} \frac{b_k}{2^k}
	= \overset{\infty}{\underset{k = 1}{\sum}} \frac{b_k'}{2^k}
	= F(x')$$
and so $F$ is well-defined.\\

To see that $F$ is continuous, let $\varepsilon > 0$ be given and $x, x' \in \mathcal{C}$ so that $|F(x) - F(x')| < \varepsilon$.

Consider the binary expansion of $\varepsilon$. Consider $\delta > 0$ such that $\delta_k = 2 \varepsilon_k$ for all $k$. Let $N$ be the first nonzero digit of $\delta$ and $\varepsilon$. Then, $|x - x'| < \delta$ implies that the first $N - 1$ digits of $x$ and $x'$ agree.

Hence, the first $N - 1$ digits of $F(x)$ and $F(x')$ agree, and so $|F(x) - F(x')| < \varepsilon$. Therefore, $F$ is continuous.\\

By the construction in part (a), we know that $0$ is represented in ternary form by always choosing the leftmost subinterval, and so for $x = 0, b_k = \frac{0}{2} = 0$ for all $k$.

Similarly, $1$ is represented in ternary form by always chosing the rightmost subinterval, and so for $x = 1, b_k = \frac{2}{2} = 1$ for all $k$. Hence
	$$F(0)
	= \overset{\infty}{\underset{k = 1}{\sum}} \frac{0}{2^k}
	= 0,$$
	$$F(1) = \overset{\infty}{\underset{k = 1}{\sum}} \frac{1}{2^k}
	= \frac{1}{2} \overset{\infty}{\underset{k = 0}{\sum}} \frac{1}{2^k}
	= \frac{\frac{1}{2}}{1 - \frac{1}{2}}
	= 1.$$\\\\


\begin{flushleft}
	\rule[-0.5ex]{17cm}{2pt}\\
		\textbf{Exercise 1.2(c):}\\
	\rule[1.5ex]{17cm}{0.5pt}
		Prove that $F: \mathcal{C} \to [0,1]$ is surjective, that is, for every $y \in [0,1]$ there exists $x \in \mathcal{C}$ such that $F(x) = y$.
	\rule[1.0ex]{17cm}{0.5pt}\
\end{flushleft}

\textbf{\textit{Proof.}}

Let $y \in [0,1]$. Then $y$ has a corresponding binary expansion.

Let $b_k$ denote the $k$-th digit of this expansion.

Construct a string $s$ such that $s_k = 2b_k$ for all $k$, where $s_k$ denotes the $k$-th digit of $s$.  This construction uniquely identifies some ternary string using only $0$ and $2$. From part (a), we know that $s$ corresponding uniquely to some $x \in \mathcal{C}$. Now, it is clear from our construction of $x$ that $F(x) = y$.\\\\

\begin{flushleft}
	\rule[-0.5ex]{17cm}{2pt}\\
		\textbf{Exercise 1.2(d):}\\
	\rule[1.5ex]{17cm}{0.5pt}
		One can also extend F to be a continuous function on $[0, 1]$ as follows. Note that if $(a, b)$ is an open interval of the complement of $\mathcal{C}$, then $F(a) = F(b)$. Hence we may define $F$ to have the constant value $F(a)$ in that interval.
	\rule[1.0ex]{17cm}{0.5pt}\
\end{flushleft}

\textbf{\textit{Proof.}}

A connected component of the complement of $\mathcal{C}$ is of the form
	$$\left( \overset{n}{\underset{i = 1}{\sum}} a_i 3^{-i} + 3^{-n}, \overset{n}{\underset{i = 1}{\sum}} a_i 3^{-i} + 2 \cdot 3^{-n}  \right)$$

for some $a_1, \dots, a_n \in \{0,2\}$.

Write $r = \overset{n}{\underset{i = 1}{\sum}} a_i3^{-i} + 3^{-n}$ so the interval is $(r, r + 3^{-n})$.

Note that
	$$r = \overset{n}{\underset{i = 1}{\sum}} a_i 3^{-i} + \overset{\infty}{\underset{i = n+1}{\sum}} 2 \cdot 3^{-i} \in \mathcal{C}$$

and
	$$\begin{aligned}
	F(r)
	&= \overset{n}{\underset{i = 1}{\sum}} \left( \frac{a_i}{2} \right) 2^{-i} + \overset{\infty}{\underset{i = n+1}{\sum}} 2^{-i}
	= \overset{n}{\underset{i = 1}{\sum}} \left( \frac{a_i}{2} \right) 2^{-i} + 2^{-n}\\
	&= \overset{n - 1}{\underset{i = 1}{\sum}} \left( \frac{a_i}{2} \right) 2^{-i} + \frac{a_n + 2}{2} \cdot 2^{-n}
	= F\left( \overset{n - 1}{\underset{i = 1}{\sum}} a_i 3^{-i} + (a_n + 2) 3^{-n} \right)\\
	&= F\left( \overset{n}{\underset{i = 1}{\sum}} a_i 3^{-i} + 2 \cdot 3^{-n} \right)
	= F(r + 3^{-n})
	\end{aligned}$$

as desired.\\\\


\begin{flushleft}
	\rule[-0.5ex]{17cm}{2pt}\\
		\textbf{Exercise 1.4(a):}\\
	\rule[1.5ex]{17cm}{0.5pt}
		\textbf{Cantor-like sets.}

		 Construct a closed set $\hat{\mathcal{C}}$ so that at the $k$-th stage of the construction one removes $2^{k-1}$ centrally situated open intervals each of length $l_k$, with
		 	$$l_1 + 2 l_2 + \cdots + 2^{k-1} l_k < 1.$$

		 If $l_j$ are chosen small enough, then $\overset{\infty}{\underset{k = 1}{\sum}} 2^{k-1} l_k < 1$.  In this case, show that $m(\hat{\mathcal{C}}) > 0$, and in fact, $m(\hat{\mathcal{C}}) = 1 - \overset{\infty}{\underset{k = 1}{\sum}} 2^{k-1} l_k$.
	\rule[1.0ex]{17cm}{0.5pt}\
\end{flushleft}

\textbf{\textit{Proof.}}

We begin by showing that we can choose the $l_j$ such that $\overset{\infty}{\underset{k = 0}{\sum}} 2^{k-1} l_k < 1$. This is clear if we choose $l_k \leq (2 + \varepsilon)^{-(k-1)}$ for any $\varepsilon > 0$ because then
	$$\overset{\infty}{\underset{k = 0}{\sum}} 2^{k-1} l_k
	\leq \overset{\infty}{\underset{k = 0}{\sum}} \left( \frac{2}{2 + \varepsilon} \right)^{k-1}$$

and the sum on the right converges and is less than $1$ for $l_0 < \frac{1}{2}$. Next, we follow a process similar to the construction of the Cantor set in defining
	$$\hat{\mathcal{C}} = \overset{\infty}{\underset{k = 0}{\cap}} I_k$$

where $I_0 = [0,1],\ I_1 = [0, \frac{1 - l_0}{2}] \cup [\frac{1 + l_0}{2}, 1]$ and each subsequent $I_k$ is obtained by taking each of the pieces in the union of $I_{k-1}$ and removing the middle $l_k$. As before, repeating this procedure yields a sequence of nested compact sets $I_0 \supset I_1 \supset I_2 \supset \cdots$ and their intersection $\hat{\mathcal{C}} \neq \emptyset$. It then follows that $\hat{\mathcal{C}}$ is measurable.\\

To find its measure, we instead compute the measure of $\hat{\mathcal{C}}^c$, which is also measurable because it's complement of a measurable set. We can see that
	$$\hat{\mathcal{C}}^c = \overset{\infty}{\underset{k = 0}{\cup}} I_k^c.$$

The complement of each $I_k$ is precisely the $2^{k-1}$ intervals of length $l_k$. Hence, themeasure of each of these $m(I_k) = 2^{k-1} l_k$. And so
	$$m(\hat{\mathcal{C}}^c)
	= m\left( \overset{\infty}{\underset{k = 0}{\cup}} I_k^c \right)
	= \overset{\infty}{\underset{k = 0}{\sum}} m(I_k^c)
	= \overset{\infty}{\underset{k = 0}{\sum}} 2^{k-1} l_k.$$

Note that $[0,1] = \hat{\mathcal{C}} \cup \hat{\mathcal{C}}^c$ and so
	$$m(\hat{\mathcal{C}}) = 1 - \overset{\infty}{\underset{k = 0}{\sum}}2^{k-1} l_k.$$\\\\


\begin{flushleft}
	\rule[-0.5ex]{17cm}{2pt}\\
		\textbf{Exercise 1.4(b):}\\
	\rule[1.5ex]{17cm}{0.5pt}
		Show that if $x \in \hat{\mathcal{C}}$, then there exists a sequence of points $\{x_n\}_{n = 1}^\infty$ such that $x_n \notin \hat{\mathcal{C}}$, yet $x_n \to x$ and $x_n \in I_n$, where $I_n$ is a sub-interval in the complement of $\hat{\mathcal{C}}$ with $|I_n| \to 0$.
	\rule[1.0ex]{17cm}{0.5pt}\
\end{flushleft}

\textbf{\textit{Proof.}}

Observe first that since $\overset{\infty}{\underset{k = 1}{\sum}} 2^{k-1} l_k < 1$, the tail of the series must go to zero. That is, for any $\varepsilon > 0$, there exists $N$ such that $l_n < \varepsilon$ for all $n \geq N$.\\

Now let $x \in \hat{\mathcal{C}}$. Let $\hat{\mathcal{C}}_k$ denote the $k$ stage of the construction.\\

For each $k, x$ belongs to some closed subset $S_k$ of $\hat{\mathcal{C}}_k$. Let $I_k$ be the open interval removed from $S_k$ to proceed to the next step of the construction.\\

We take any $x_k \in I_k$ to form our sequence $\{x_n\}_{n = 1}^\infty$. Clearly, each $x_k$ belongs to an sub-interval in the complement of $\hat{\mathcal{C}}$. Furthermore, $|I_k| = l_k \to 0$. It remains to show that $x_n \to x$.\\

From the construction of $\hat{\mathcal{C}}_k$ and our selection of $x_n$, it is clear that
	$$|x - x_n| < |I_n| + |S_n|.$$

By our previous observation, we know that $|I_n| = l_n \to 0$. Now
	$$|S_n| - \frac{1 - \overset{n}{\underset{k = 1}{\sum}} 2^{k-1} l_k}{2^n}
	\leq \frac{1}{2^n} \to 0
	\quad \text{as } n \to \infty.$$

Hence, $|x - x_n| \to 0$. That is, $\{x_n\}_{n=1}^\infty$ converges to $x$.\\\\\\



\begin{flushleft}
	\rule[-0.5ex]{17cm}{2pt}\\
		\textbf{Exercise 1.4(c):}\\
	\rule[1.5ex]{17cm}{0.5pt}
		Prove as a consequence that $\hat{\mathcal{C}}$ is perfect, and contains no open interval.
	\rule[1.0ex]{17cm}{0.5pt}\
\end{flushleft}

\textbf{\textit{Proof.}}

\begin{enumerate}

	\item \textbf{Perfect:}

		let $\varepsilon > 0$ be given and consider $B(x, \varepsilon)$ for any $x \in \hat{\mathcal{C}}$. We can find $N \in \mathbb{N}$ such that $S_N \subset B(x, \varepsilon)$.\\

		Now, this interval must have two endpoints $a_N$ and $b_N$. By the construction of $\hat{\mathcal{C}}$, we know that the endpoints of any interval are never removed, and so $a_N, b_N \in \hat{\mathcal{C}}$. Furthermore, we have that $a_N, b_N \in S_N \subset B(x, \varepsilon)$. Therefore, $x$ is not isolated.\\

	\item \textbf{No open interval:}

		We try to prove by contradiction. Suppose that there exists an open interval $O \in \hat{\mathcal{C}}$. Then, for any $x \in O$, there exists $\varepsilon_0$ such that $B(x, \varepsilon_0) \subseteq O$.\\

		Let $\varepsilon < \varepsilon_0$. Then, there can be no sequence $\{x_n\}_{n=1}^\infty$ of the type described in part (b) whose limit is $x$, since $B(x, \varepsilon_0) \subseteq \hat{\mathcal{C}}$ implies that $|x - x_n| > \varepsilon_0 > \varepsilon$ for all $n$.\\

		This contradicts the conclusion of part (b), and so it must be that $\hat{\mathcal{C}}$ contains no open interval.\\\\


\end{enumerate}

\begin{flushleft}
	\rule[-0.5ex]{17cm}{2pt}\\
		\textbf{Exercise 1.4(d):}\\
	\rule[1.5ex]{17cm}{0.5pt}
		Show also that $\hat{\mathcal{C}}$ is uncountable.
	\rule[1.0ex]{17cm}{0.5pt}\
\end{flushleft}

\textbf{\textit{Proof.}}

We try to prove by contradiction. Suppose that $\hat{\mathcal{C}}$ is countable and let $I_n$ be an enumeration of the set.

$I_1$ belongs to exactly one of the two intervals in $\hat{\mathcal{C}}_1$, denote the interval which fails to contain $I_1$ by $F_1$.

In $\hat{\mathcal{C}}_2$, 2 disjoint intervals $\subseteq F_1$, one of them say $F_2$ must fail to contain $I_2$.

Repeating the process, we have decreasing sequence of closed interval $F_n$ of length $f_n$ s.t. $I_n$ not belongs to $F_n$ and
	$$\emptyset \neq \overset{\infty}{\underset{n = 1}{\cap}} F_n \subseteq S.$$

Hence
	$$\exists I \in \hat{\mathcal{C}}
	\quad \text{and} \quad I \neq I_n,\ \forall n \in \mathbb{N}.$$

We have a contradiction.


%%%%%%%%%%%%%%%%%%%%%%%%%%%%%%%%%%%%%%%%%%%%%%%%%%%%%%%%%%%%%%%%%%%%%%%%%%%
	% \begin{flushleft}
	% 	\rule[-0.5ex]{17cm}{2pt}\\
	% 		\textbf{Theorem}\\
	% 	\rule[1.5ex]{17cm}{0.5pt}
			
	% 	\rule[1.0ex]{17cm}{0.5pt}\
	% \end{flushleft}
	% \textbf{\textit{Proof.}}\\

\end{document}






